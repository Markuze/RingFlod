%%%%%%%%%%%%%%%%%%%%%%%%%%%%%%%%%%%%%%%%%%%%%%%%%%%%%%%%%%%%%%%%%%%%%%%%%%%%%%%%
% Template for USENIX papers.
%
% History:
%
% - TEMPLATE for Usenix papers, specifically to meet requirements of
%   USENIX '05. originally a template for producing IEEE-format
%   articles using LaTeX. written by Matthew Ward, CS Department,
%   Worcester Polytechnic Institute. adapted by David Beazley for his
%   excellent SWIG paper in Proceedings, Tcl 96. turned into a
%   smartass generic template by De Clarke, with thanks to both the
%   above pioneers. Use at your own risk. Complaints to /dev/null.
%   Make it two column with no page numbering, default is 10 point.
%
% - Munged by Fred Douglis <douglis@research.att.com> 10/97 to
%   separate the .sty file from the LaTeX source template, so that
%   people can more easily include the .sty file into an existing
%   document. Also changed to more closely follow the style guidelines
%   as represented by the Word sample file.
%
% - Note that since 2010, USENIX does not require endnotes. If you
%   want foot of page notes, don't include the endnotes package in the
%   usepackage command, below.
% - This version uses the latex2e styles, not the very ancient 2.09
%   stuff.
%
% - Updated July 2018: Text block size changed from 6.5" to 7"
%
% - Updated Dec 2018 for ATC'19:
%
%   * Revised text to pass HotCRP's auto-formatting check, with
%     hotcrp.settings.submission_form.body_font_size=10pt, and
%     hotcrp.settings.submission_form.line_height=12pt
%
%   * Switched from \endnote-s to \footnote-s to match Usenix's policy.
%
%   * \section* => \begin{abstract} ... \end{abstract}
%
%   * Make template self-contained in terms of bibtex entires, to allow
%     this file to be compiled. (And changing refs style to 'plain'.)
%
%   * Make template self-contained in terms of figures, to
%     allow this file to be compiled. 
%
%   * Added packages for hyperref, embedding fonts, and improving
%     appearance.
%   
%   * Removed outdated text.
%
%%%%%%%%%%%%%%%%%%%%%%%%%%%%%%%%%%%%%%%%%%%%%%%%%%%%%%%%%%%%%%%%%%%%%%%%%%%%%%%%

\documentclass[letterpaper,twocolumn,10pt]{article}
\usepackage{usenix2019_v3}

% to be able to draw some self-contained figs
\usepackage{tikz}
\usepackage{amsmath}

% inlined bib file
\usepackage{filecontents}

%-------------------------------------------------------------------------------
\begin{document}
%-------------------------------------------------------------------------------

%don't want date printed
\date{}

% make title bold and 14 pt font (Latex default is non-bold, 16 pt)
\title{\Large \bf Bypassing IOMMU via the Linux Network stack}

%for single author (just remove % characters)
\author{
{\rm Markuze Alex}\\
Technion, VMware Research
\and
{\rm Gil Kupfer}\\
Technion
% copy the following lines to add more authors
\and
{\rm Nadav Amit}\\
VMware Research
\and{\rm Dan Tsafrir}\\
Technion, VMware Research
} % end author

\maketitle

%-------------------------------------------------------------------------------
\begin{abstract}
%-------------------------------------------------------------------------------
Your abstract text goes here. Just a few facts. Whet our appetites.
Not more than 200 words, if possible, and preferably closer to 150.
\end{abstract}


%-------------------------------------------------------------------------------
\section{Introduction}
%-------------------------------------------------------------------------------
Our contributions:
\subsection{All contributions made in Gils work \cite{gil}}.

\begin{description}
  
\item[Sub Page] Exploiting the sub page vulnerability to access metadata.
\item[Deferred protection] Exploiting IOTLB invalidation to hack correct NIC drivers.
\item[KASLR] Reading random ptrs to guess the random address bits, Leaking data via Sub Page.
\item[FreeBSD UMA] Hacking FreeBSD memory allocator.
\item[\texttt{Question RDMA}] Can IPoIB driver be exploited in additional ways? Other RDMA issues.
\end{description}

\subsection{All contributions made in NDSS paper\cite{thunder}:}
\begin{description}
\item[attack 1,2] Windows attacks, irrelevant.
\item[attack 3] MAC OS reading VPN data, irrelevant.
\item[attack 4] Shared mappings between devices: MAC OS (Also Windows)mbufs open to read/write - Same as Gils  attack \#1 on thunderbolt.
\item[attack 5] FreeBSD per device iommu:  Same as Gils  attack \#1 on thunderbolt (p.9). Mention that Linux is invulnerable.
\item[attack 6] Linux leaks nat tables - \texttt{Seems half baked, but needs farther investigation.}
\item[attack 7] Linux leaking data by avoiding tx completions - would hit T/O issues and driver restart.
\item[attack 8] ATS - Self serve IOMMU by NIC. 
\end{description}

\subsection{Shared contributions:}
\begin{description}
\item[Sub Page] Shared Mdata - Data mapping.
\item[KALSR] KASLR leaking due to sub page.
\end{description}

\subsection{New Contribution}

\begin{description}
\item[skb shared info] Shared Mdata in Linux skb. 
\item[page cahe colocation] RX data allocated from page cache - means that after unmap page is still accessible via iova of next buffer - (strict and deferred vulnerable). This is stronger then sub page, as dense mappings 
allow for an attack even with strict API and correct ordering.
\item[RingFlod] Determining PFNs for attack. \texttt{What else is accessible by the NIC, what config, FW return codes/values? - Is kva visible}.
\item[HOTPage] A TX page reused in RX - allowing for temp R/W access with open valid kva (zero knowledge)-- \texttt{I'm pretty sure overlap is impossible and this approach is invalid}.
\end{description}

\newpage

\bibliographystyle{plain}
\bibliography{references}

%%%%%%%%%%%%%%%%%%%%%%%%%%%%%%%%%%%%%%%%%%%%%%%%%%%%%%%%%%%%%%%%%%%%%%%%%%%%%%%%
\end{document}
%%%%%%%%%%%%%%%%%%%%%%%%%%%%%%%%%%%%%%%%%%%%%%%%%%%%%%%%%%%%%%%%%%%%%%%%%%%%%%%%

%%  LocalWords:  endnotes includegraphics fread ptr nobj noindent
%%  LocalWords:  pdflatex acks