
\section{\textcolor{red}{Applicability to other OSs}}\label{sec:other_os}

%\begin{table*}[t]
%\begin{adjustbox}{max width=0.9\textwidth}
%\begin{tabular}{l|ccccccc}
%OS                           & default mode & IOVA map   & memory isolation    & KASLR &  Known Vulnerability & Fix in later versions \\ \hline

%Windows 10  & \V   & \V & \X & \V   & \textcolor{red}{Yes}       & partial   \\
%MacOS       & \V   & \X & \X & \V   & \textcolor{red}{Yes}       & partial   \\
%FreeBSD     & \X   & \V & \X & \V   & \textcolor{red}{Yes}       & No   \\
%Linux       & \V   & \V & \X & \V   & \textcolor{green}{No}      &  \\\hline       
%\end{tabular}
%\end{adjustbox}
%  \caption{IOMMU adaptation in different OS's.}
%  \label{tab:other_os}
%\end{table*}

%\begin{table*}[t]
%\begin{adjustbox}{max width=0.9\textwidth}
%\begin{tabular}{l|ccccccc}
%OS                           & default mode & IOVA map   & memory isolation    & KASLR &  DMA Vulnerability & previously unknown          \\ \hline

%Windows 10  & \V   & \V & \X & \V   & likely        &    \\
%MacOS       & \V   & \X & \X & \V   & likely        &    \\
%FreeBSD     & \X   & \V & \X & \V   & demonstrated  &    \\
%Linux       & \V   & \V & \X & \V   & compound      & \V \\\hline       
%MacOS \textless 10.12.4      & supported     & On       & shared     & dedicated & Yes   & demonstrated             \\
%Windows 10 \textless 1803    & Not supported & N/A          & N/A        & shared    & Yes   & demonstrated         \\
%MINIX 3                      & supported     & On       & per device          & dedicated & No    & unknown                  \\\hline
%Linux                        & supported     & On     & per device & shared    & Yes   & \textbf{\begin{tabular}[c]{@{}c@{}}demonstrated in this paper \\ (previously unknown)\end{tabular}} \\\hline       
%\end{tabular}
%\end{adjustbox}
%  \caption{IOMMU adaptation in different OS's.}
%  \label{tab:other_os}
%\end{table*}

\adam{strange title. Perhaps call it: Applicability to Non-Linux OSs}\SV{Better?}
The current state of IOMMU adaptation varies among different OS vendors. We briefly discuss other OSes below.

\smallskip
\noindent\textbf{Windows.} Until recently (2019) Windows had no IOMMU support, exposing it to simple DMA attack.
In 2019 (with build 1803) Microsoft has introduced Kernel DMA Protection \cite{ms_iommu}, which provides IOMMU protection by default with a dedicated I/O page table per device. 
In addition, network buffers are allocated from dedicated pools of memory, which limits the possible exposure of sensitive data. But a brief exploration of the Windows Networking drivers API reveals functions like \emph{NdisAllocateNetBufferMdlAndData} \cite{ms_single} that allocates a NET\_BUFFER structure and data in a single memory buffer, exposing the OS to \simple attacks. 
Note that the NET\_BUFFER vulnerability was previously described\cite{thunder}. 

\smallskip
\noindent\textbf{MacOS.} IOMMU is enabled by default. It also uses dedicated memory for network I/O. MacOS, however, does expose the \textit{mbuf} data structure to the device, though with some precautions such as blinding the exposed callback pointer \textit{ext\_free} by XORing it with a secret cookie.
Indeed, this is sufficient to defend against \simple attacks. However, such an exposure of metadata opens up the MacOS to potential \compound attacks.
%While stopping \simple attacks, such an exposure of metadata opens up the MacOS to potential \compound attacks. 
\textcolor{violet}{In particular, while the value of the secret cookie is random, \textit{ext\_free} can receive only one of two possible values. As a results, once an attacker compromises macOS KASLR (as demonstrated in \cite{thunder}), the random cookie is revealed by a single XOR operation.}  
\adam{how, if pointers are blinded? Explain.}\SV{let's discuss}

\smallskip
\noindent\textbf{FreeBSD.} An \textit{mbuf} struct which is used for networking exposes the \textit{ext\_free} callback pointer. An attack on FreeBSD via this callback pointer was  demonstrated by Merketos\cite{thunder}. To the best of our knowledge (October 2020), this vulnerability still exists in the FreeBSD kernel. 

%\smallskip
%\noindent\textbf{Minix.} An honorable mention\adam{is this cynical? if so, remove it. if not, not clear why an honorable mention goes to an OS that in the next section we're told is vulnerable} goes to the Minix micro-kernel\cite{minix} which is present on Intel chipsets. It has been shown to have a DMA vulnerability at initialization time\cite{minix_intc}. The MINIX micro-kernel pre-allocates memory for I/O and copies in/out any sent/received data bytes, effectively stopping known DMA attacks. 
%\adam{suggest removing Minix. who cares?}
