\subsection{DMA Kernel Address SANitizer}\label{sec:dma-kasan} 

In \DIFdelbegin \DIFdel{Sec.}\DIFdelend \DIFaddbegin \DIFadd{Section}\DIFaddend ~\ref{sec:static-analysis}, we \DIFdelbegin \DIFdel{have shown }\DIFdelend \DIFaddbegin \DIFadd{demonstrated }\DIFaddend that more than 70\% of DMA-map operations result in exposed pointers. 
%Exposed pointers may be used in a DMA attack either passively to subvert KASLR or actively to execute an attack. 
Most of the remaining 30\% \DIFaddbegin \DIFadd{of }\DIFaddend DMA-map operations are executed on allocated objects that are presumably not co-located on the same page with vulnerable \DIFdelbegin \DIFdel{meta-data}\DIFdelend \DIFaddbegin \DIFadd{metadata}\DIFaddend . However, this is often not the case in practice.
Indeed, objects allocated via the kmalloc API~\cite{Cor07} may share a page with objects of similar size. As a result, vulnerable metadata may still be mapped. 
%
Such a vulnerability is not visible to \tool as it is of a dynamic nature. Accordingly, we \DIFdelbegin \DIFdel{have }\DIFdelend developed a run-time tool that reports such vulnerabilities. 

Our solution is based on an existing kernel tool, KASAN~\cite{kasan}, which is a dynamic memory error detector designed to detect out-of-bounds and use-after-free bugs. KASAN uses shadow memory to record whether a memory byte is safe to access. \DIFdelbegin \DIFdel{KASAN }\DIFdelend \DIFaddbegin \DIFadd{It also }\DIFaddend uses compile-time instrumentation to insert checks of shadow memory on each memory access. 
We \DIFdelbegin \DIFdel{modify }\DIFdelend \DIFaddbegin \DIFadd{modified }\DIFaddend KASAN to record DMA-map operations in addition to memory allocations. Our tool, \DIFdelbegin \DIFdel{termed }\DIFdelend \DIFaddbegin \DIFadd{referred to as }\DIFaddend DMA-KASAN (\dkasan)\DIFdelbegin \DIFdel{reports}\DIFdelend \DIFaddbegin \DIFadd{, reports the following}\DIFaddend : 
\begin{enumerate}
    \item alloc-after-map:  kmalloc object is allocated from a mapped page.
    \item map-after-alloc:  the containing page is mapped after an object was allocated.
    \item access-after-map: the CPU accesses a DMA mapped page.
    \item multiple-map: an object is mapped multiple times with possibly different permissions.
\end{enumerate}
%alloc-after-map, map-after-alloc.We term by ``alloc-after-map'' a situation in which kmalloc object is allocated from a mapped page. Likewise, we term by ``map-after-alloc'' a situation in which the containing page is mapped after an object was allocated.
%\adam{the meaning of these terms shouldn't be in a footnote, it's important}
%The tool also detects cases when a DMA mapped page is accessed by the CPU. 
%\dkasan, therefore, identifies all cases where an allocated object was inadvertently dma-mapped after allocation or has already been allocated on a dma-mapped page. 
We tested \dkasan using \DIFdelbegin \DIFdel{our setup (Sec.}\DIFdelend \DIFaddbegin \DIFadd{the setup described in Section}\DIFaddend ~\ref{Sec:setup}).
%\adam{setup wasn't discussed or defined until this point}\SV{Added a forward ref}.
In our experiment we cloned a large project from a \DIFdelbegin \DIFdel{git }\DIFdelend \DIFaddbegin \DIFadd{Git }\DIFaddend repository and compiled it concurrently with light network traffic (i.e., ICMP ping). This experiment \DIFdelbegin \DIFdel{has }\DIFdelend identified numerous cases where a DMA-mapped page is used to hold network and file system metadata. \DIFdelbegin \DIFdel{Example }\DIFdelend \DIFaddbegin \DIFadd{Sample }\DIFaddend results are shown in \DIFdelbegin \DIFdel{Fig.}\DIFdelend \DIFaddbegin \DIFadd{Figure}\DIFaddend ~\ref{fig:dkasan-report}. Each line shows an allocation that \DIFdelbegin \DIFdel{resulted }\DIFdelend \DIFaddbegin \DIFadd{results }\DIFaddend in a random exposure; namely, it shows the size of the allocated buffer, the DMA access type\DIFaddbegin \DIFadd{, }\DIFaddend and the allocating location (i.e., function name and offset).

\DIFdelbegin \DIFdel{Fig.}\DIFdelend \DIFaddbegin \DIFadd{Figure}\DIFaddend ~\ref{fig:dkasan-report} also shows how random kernel data structures can be mapped for both READ and WRITE. Some of these\DIFaddbegin \DIFadd{, such as line 5, }\DIFaddend also contain callback pointers\DIFdelbegin \DIFdel{(e. 
 g., line 5). 
 }\DIFdelend \DIFaddbegin \DIFadd{. 
 }\DIFaddend 

%
%In case of deferred protection
%It is important to note that even if these pages are invalidated, an access attempt by the device will %simply result in a dmesg warning line. Namely, the device may repeatedly probe pages for access.
\begin{figure}[t]
\begin{adjustbox}{width=0.9\linewidth}
\lstset{
    escapechar={|},
}
        \begin{lstlisting}[
        basicstyle = \small,
        %basicstyle=\ttfamily,
        columns = fixed,
        tabsize=1,
        %frame = l,
        language = C
        ]
[1] size 512 [|\color{purple}READ, WRITE|] __alloc_skb+0xe0/0x3f0
[2] size 512 [|\color{purple}WRITE|] load_elf_phdrs+0xbf/0x130
[3] size 512 [|\color{purple}WRITE|] __do_execve_file.isra.0+0x287/0x1080
[4] size 64  [|\color{purple}WRITE|] sock_alloc_inode+0x4f/0x120
[5] size 328 [|\color{purple}READ, WRITE|] assoc_array_insert+0xa9/0x7e0
        
\end{lstlisting}
\end{adjustbox}
        \caption{\dkasan report example\DIFdelbeginFL \DIFdelFL{.}\DIFdelendFL }
        %\vspace{-3mm}
        \label{fig:dkasan-report}
\end{figure}

%\subsection{Random Exposure Exploits}\label{sec:rand_exp}
While we do not demonstrate random exposure exploits, these findings indicate that random exposure vulnerabilities should not be disregarded. Accordingly, in \DIFdelbegin \DIFdel{Sec.~\ref{appx:additional_compound}we show }\DIFdelend \DIFaddbegin \DIFadd{Section~\ref{appx:additional_compound}, we present }\DIFaddend a compound attack that exploits an I/O buffer that is mapped twice, once for read and once for write. Line 1 in \DIFdelbegin \DIFdel{Fig.}\DIFdelend \DIFaddbegin \DIFadd{Figure}\DIFaddend ~\ref{fig:dkasan-report}  shows how such double mapping innocently occurs.

\subsection{Discussion and Limitations}
\dkasan is a run-time tool that has a large memory footprint and the obvious overhead of callbacks on each memory access. This tool is useful for testing specific systems for vulnerabilities.
\tool is a static analysis tool that may fail to follow a mapped variable due to potential code obfuscation \DIFdelbegin \DIFdel{like }\DIFdelend \DIFaddbegin \DIFadd{such as }\DIFaddend function pointers, macros\DIFaddbegin \DIFadd{, }\DIFaddend and others, potentially resulting in a false-negative result. False positives may happen in \DIFdelbegin \DIFdel{a rare case
}\DIFdelend \DIFaddbegin \DIFadd{the rare situation 
}\DIFaddend %\adam{the footnote and its purposes are't clear. Bugs by who\textcolor{red}{\textbf{m}}? Tool writers (you) or kernel devs? And what exactly is the point?}\SV{Removed} 
where the mapped data structure crosses a  page boundary. In this case, \tool may flag a callback function \DIFdelbegin \DIFdel{, which }\DIFdelend \DIFaddbegin \DIFadd{that }\DIFaddend may not be exposed, since it resides on a different page. \DIFdelbegin \DIFdel{Namely, only }\DIFdelend \DIFaddbegin \DIFadd{Only }\DIFaddend part of a data structure is accessible to the device due to the \subpage{} vulnerability at the mapped page, whereas the callback pointer resides on a different page \DIFdelbegin \DIFdel{which }\DIFdelend \DIFaddbegin \DIFadd{that }\DIFaddend is not accessible to the device.%\adam{explain why this can happen}\SV{Is this what you were looking for?}

Our code for \dkasan is publicly available \cite{DKASAN}.