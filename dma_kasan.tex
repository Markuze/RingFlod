\section{\dkasan}\label{sec:dma-kasan} 

In section Sec. \ref{sec:static-analysis}, we have shown that more than 60\% of dma-map operations result in exposed pointers. Exposed pointers may be used in a DMA attack either passively to subvert KASLR or actively to execute an attack. Most of the remaining 40\% dma-map operations are executed on allocated objects that are by design, presumably, not co-located on the same page with vulnerable meta-data. However, this is often not the case in practice.
Indeed, objects allocated via the kmalloc API~\cite{Cor07}, may share a page with objects of similar size. As a result vulnerable meta-data may still be inadvertently mapped. 
%
Such a vulnerability is not visible to \tool as it is of a random access nature. Accordingly, we have developed a run-time tool that reports such vulnerabilities. 

Our solution is based on an existing kernel tool, KASAN~\cite{kasan}, which is a dynamic memory error detector designed to detect out-of-bound and use-after-free bugs. KASAN uses shadow memory to record whether a memory byte is safe to access. KASAN uses compile-time instrumentation to insert checks of shadow memory on each memory access. 
We modify KASAN to record DMA-map operations in addition to memory allocations. Our tool, termed DMA-KASAN (\dkasan) reports alloc-after-map, map-after-alloc\footnote{We term by ``alloc-after-map'' a situation in which kmalloc object is allocated from a mapped page. Likewise, we term by ``map-after-alloc'' is a situation in which the containing page is mapped after an object was allocated.}, and cases when a DMA mapped page is accessed by the CPU. \dkasan, therefore, identifies all cases where an allocated object was inadvertently dma-mapped after allocation or has already been allocated on a dma-mapped page. We tested \dkasan using our setup. In our experiment we cloned a large project from a git repository and compiled it concurrently with light network traffic (i.e., ICMP ping). In this experiment, we have identified numerous cases where a dma-mapped page is used to allocate file system metadata; example results are shown in Fig. \ref{fig:dkasan-report}. In these cases pages that were used for RX were promptly re-purposed for kmalloc and were still visible to the device\footnote{Due to deferred protection policy.}. An attempt to access an invalidated \iova will simply result in a dmesg warning line.

\begin{figure}[t]
\begin{adjustbox}{width=\linewidth}
        \begin{lstlisting}[
        basicstyle = \small,
        %basicstyle=\ttfamily,
        columns = fixed,
        tabsize=1,
        %frame = l,
        language = C
        ]
$ size 48 [READ] ext4_ext_remove_space+0x881/0x2340
$ size 48 [WRITE] ext4_readdir+0x103a/0x1520
$ size 64 [WRITE] ext4_htree_store_dirent+0x60/0x200
$ size 64 [WRITE] sock_alloc_inode+0x4f/0x120
        \end{lstlisting}
\end{adjustbox}
        \caption{\dkasan report example}
        \label{fig:dkasan-report}
\end{figure}
