\section{\dkasan}\label{sec:dma-kasan}

In section Sec. \ref{sec:static-analysis}, we have shown that more than 60\%of dma-map operations result in exposed pointers. Exposed pointers may be used in a DMA attack either passively to break KASLR or actively to execute an attack. Most of the other 40\% dma-map operations were executed on allocated objects that are not co-located on the same page as vulnerable meta-data. Presumably making the mapping safe, but often objects allocated via the kmalloc API share a page with objects of similar size. The implication of this is that even, if for example, the firewire driver (Sec. \ref{sec:sbp2_attack}) did not map inline orb objects but rather allocated them via kmalloc, these similarly sized objects would occupy the same physical page resulting in a high probability attack. Such an attack is not visible to our \tool analysis as it is a random access vulnerability. To identify random access vulnerabilities, we have developed a run-time tool that reports such issues. 

We base our DMA-Kernel-Address-SANitizer(\dkasan) on an existing kernel tool, KernelAddressSANitizer(KASAN)\cite{kasan} which is a dynamic memory error detector designed to find out-of-bound and use-after-free bugs. KASAN uses shadow memory to record whether a memory byte is safe to access. KASAN uses compile-time instrumentation to insert checks of shadow memory on each memory access. We modify KASAN to record DMA-mapped operations in addition to memory allocations. D-KASAN reports on allocation-after-map, map- after-allocation, and cases when a DMA mapped page is accessed by the CPU. \dkasan, therefore, identifies all cases where an allocated object was inadvertently dma-mapped sometime after allocation or has already been allocated on a dma-mapped page. We tested \dkasan on our setup, in our experiment we cloned a Linux kernel from git repository and compiled it concurrently with light network traffic (i.e., ICMP ping). In this short experiment, we have identified numerous cases where a dma-mapped page is used to allocate file system metadata; example results are shown in Fig. 12. In this cases pages that were used for RX were promptly re-purposed for kmalloc and were still visible to the device\footnote{Due to deferred protection policy}. An attempt to access an invalidated \iova will simply result in a dmesg warning line

\begin{figure}[t]

        \begin{lstlisting}[
        basicstyle = \small,
        %basicstyle=\ttfamily,
        columns = fixed,
        tabsize=1,
        %frame = l,
        language = C
        ]
$ size 48 [WRITE] ext4_ext_remove_space+0x881/0x2340
$ size 48 [WRITE] ext4_readdir+0x103a/0x1520
$ size 64 [WRITE] ext4_htree_store_dirent+0x60/0x200
$ size 64 [WRITE] sock_alloc_inode+0x4f/0x120
                \end{lstlisting}
        \caption{\dkasan report example}
        \label{fig:dkasan-report}

\end{figure}
