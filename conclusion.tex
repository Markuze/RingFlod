\section{IPoIB/RDMA? - DPDK, NetMap?}
We didnt find any risks associated with RDMA other than the methods listed in this paper, the ipoib driver also maps the \shinfo. But a malicious device is still needed, as the post\_send/post\_receive API is simmilar in function to the usual way NICs function; namely a remote user can pick where to write or to write more than once to the same address intentionally. For these kinds of attacks rdma\_read/write API is needed, which provides the peer the ability to read/write to a chosen addresses. We didn't find any uses for this API in the Linux kernel. \textcolor{red}{AFAIK; besides Linux only Windows supports RDMA in the kernel}
\section{Related Work}
\textcolor{blue}{Page 13, when all else is done, may spend some lines to rip into the ndss paper}
\section{Conclusion}

Hardware attacks are often considered to be harder to implement than software attacks. Nevertheless, once a malicious device is built, launching the attack is as easy as connecting the device to an external port for only a few seconds. Recent leaks from clandestine agencies show that they attacked both by shipping infected hardware \cite{Gal14} and by connecting external malicious devices \cite{Fin14}. Hence, the problems really do exist in the wild and should concern security experts. Moreover, our FireWire/SPB2 attack can be seen as an expansion of the classic attacks and could easily be added as a module to existing off-the-shelf attacking tools.
The main limitation of our attacks compared to the old ones is that the devices in our attacks have access only to part of the memory. For instance, memory dumping, a popular DMA attack, has become much less trivial thanks to the IOMMU. In the extreme case, the attacker has access only to one page without even knowing its physical address. As a result, our attacks had to be made more complicated than the old ones. For example, code injection might be required for memory dumping, and payload spraying for overcoming unknown physical addresses.
Generally speaking, the requirement for physical access means that the attacks are targeted albeit this need not be true. There are some methods for spreading opportunistic attacks using physical devices. In some case, modifying firmware to be malicious could be done entirely by software that, in its turn, could be distributed over the network (e.g., Bad USB attack \cite{NL14}). In some cases, the attack could even come from the network side of the device [Ben17a, Ben17b]. Another method is to drop malicious devices in public spaces and wait for a potential victim to connect it to his computer \cite{TDF16}.
While working on the attacks, we realized that the root cause of all the discussed problems comes from the way modern OSs treat peripheral devices. Today, the se- curity industry no longer trusts I/O devices. Recommendations for building trusted environments often include putting limitations on devices, such as removing potentially dangerous drivers or using IOMMU. All state-of-the-art OSs, however, were designed a long time ago. Indeed, when it comes to security, OSs do not treat devices as untrusted entities as they do processes. All the lessons that were learned about protection from/of processes do not seem to have carried over. As a result: (1) Memory that is given to the device is not zeroed; (2) Allocation granularity is not page aligned; (3) IOTLB invalidations can be deferred; (4) IOVAs are not randomized and are predictable; and (5) There are no checkers for invalid memory use (such as Linux’s KASAN for regular memory usages). Even though we explored only some aspects of this problem, this is a fundamental design issue and it should be addressed as such. Finally, we note that current IOMMU/PCI architectures allow peripheral devices to serve as their own IOTLB (namely, PCI ATS or Device-IOTLB). This technology makes our work meaningless because, by using it, malicious devices are able to simply report false translations and access any protected memory. To somewhat enhance the security of Linux, we suggested a patch allowing system administrators to disable ATS.1 Development of this patch continues and it will most likely be introduced by Linux into one of its upcoming versions.\newline
DMA attacks are feasible and should not be treated lightly. IOMMU subverting attacks are avoidable, but none of the solutions can be found in the wild. Better kASLR, NX-BIT and better API are all needed to prevent IOMMU subverting techniques. 
\textcolor{olive}{use elsewhere:Its important to note that no DMA attack could work in the presence of IOMMU and the absence of sub-page vulnerability.}

\subsubsection{Mitigations in the wild}
\textcolor{magenta}{I imagine a table with OS on Y and best practices on X.
IOMMU policy, KASLR, NX-bit, discriminate mapping (R or W, not both), device IOVA separation, sand boxing mapped addresses.
Kernels Win,MacOS,FreeBSD - use NDSS paper, contribution: ESX (Need to send some emails), Linux - Ubuntu versions, Sless?(RHEL)}.