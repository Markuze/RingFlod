

\section{Discussion and conclusions}

Hardware attacks are often considered to be harder to implement than software attacks. Nevertheless, once a malicious device is built, launching the attack boils down to connecting the device to an external port and for only a few seconds. Furthermore, recent leaks from clandestine agencies show that they have been attacked both by shipping infected hardware \cite{Gal14} and by connecting external malicious devices \cite{Fin14}. 

\begin{comment}
\footnote{\url{https://lore.kernel.org/lkml/20180510230948.GF190385@bhelgaas-glaptop.roam.corp.google.com/}}.
\end{comment}

Clearly, the root cause, for DMA attacks is sub-page vulnerability. Namely, no DMA attack could work without the unintentional exposure of restricted fields. The blame is not with the device drivers alone. With the existing API used for I/O operations and due to performance considerations, it is extremely difficult not to create a sub-page vulnerability. Thus, even well-written drivers can be subverted by the OS (e.g., bnx2 by deferred protection). Examples to this include:

\begin{enumerate}
    \item API: 
    \begin{itemize}
        \item The \textit{dma\_map\_single} call accepts a pointer and the buffer length. This API insinuates that only the mapped bytes are exposed, when, in fact, the whole page is accessible.
        \item \textit{dma\_unmap\_single}, insinuates that the buffer is not accessible to the device after the call; this does not hold both due to deferred protection and type (c) sub-page vulnerabilities.
        \item \textit{build\_skb} facilitates building an \skb{} around an arbitrary I/O buffer, in turn, embedding critical data structures inside an I/O buffer.
        \item While \texttt{page\_frag} is an efficient allocator, it inherently creates a type (c) sub-page vulnerability.
    \end{itemize} 
    \item Tools/Infrastructure: 
    \begin{itemize}
            \item \texttt{page\_pool} API, by caching I/O buffers and avoiding expensive unmaps, the drivers \emph{circumvent} the security techniques of the OS. 
            \item \shinfo{} is by design built within an I/O buffer. Avoiding type (b) sub-page vulnerabilities impose a challenge.
            \item No dedicated allocators for I/O such as proposed in previous works (e.g., \cite{MSMT18,MMT16}).
    \end{itemize}
\end{enumerate}

We contend that a better API and better mechanisms can provide driver authors with better options for writing secure and performant device drivers. Specifically, for \shinfo{}, which is ingrained in the Linux network stack, we urge the use of pointer obfuscation \cite{Coo17}, and solely non-linear RX buffers as an alternative to a new design. Additionally, we urge the use of static code analysis that detects sub-page vulnerabilities.

\emph{All DMA attacks stem from sub-page vulnerabilities - Simply stop sharing restricted data.}
