\section{Introduction}

Direct Memory Access (DMA) is a technology that allows input-output (I/O) devices to access the memory without CPU involvement; improving system performance.  
Yet, in its basic form, DMA makes the system vulnerable to DMA attacks, which are carried out by malicious devices that access memory regions not intended for their use. 

\SV{Add information regarding DMA capable devices}

DMA attacks are well-known and have existed in the wild for over a decade \cite{Dor04,BDK10,thunder}. They range from stealing and manipulating sensitive data to taking over the victim machine. Popular attacks include: opening a locked computer \cite{MM, Fin14}; executing arbitrary code on the victim machine \cite{Fri16, Woj08, AD10,thunder}; stealing sensitive data items such as passwords \cite{SB12, LKV13, Cim16, BR12}; or extracting full memory dumps of victim machines for offline analysis \cite{MM, Vol, Fin14, GA10}. 

Modern systems protect themselves against such DMA attacks by using the input-output memory management unit (IOMMU). IOMMU, inspired by the design of the ordinary MMU, adds a layer of virtual memory to devices. That is, instead of using physical addresses, the devices use I/O virtual addresses (IOVAs) which are translated into physical addresses by the IOMMU during each I/O transaction. Hence, by design, IOMMU provides address space isolation by allowing devices to access only their mapped pages, leaving all other memory pages inaccessible. 

But unlike processes that operate in page granularity, I/O buffers can be significantly smaller and share pages with other kernel buffers. For this specific reason, known as the sub-page vulnerability \cite{MMT16,thunder}, IOMMU fails to protect the kernel from unprivileged access. That is, when I/O buffers are co-located with sensitive kernel buffers in the same page, they are able to access potentially sensitive data. 

Malicious devices can exploit this vulnerability to construct their attacks. These range from simple random memory access exploits (e.g., memory corruption that can be used to crush the kernel) to the more dire privilege escalation attacks; where the attacker gains control over the victim machine.

\SV{related work exists but it only mentions specific attacks and specific solutions... there is no schema to identify potential threats...}

Accordingly, in this work we present a schema for identifying viable attack vectors by malicious DMA capable devices. Specifically, we introduce \means,\motivation,\oportunity (MMO) a trifecta of attributes, (Section \ref{sec:mmo}):
\begin{enumerate}
    \item \motivation: A kernel buffer filled with malicious code; a \mabaf.
    \item \means: The kernel virtual address (\kva) of a \mabaf.
    \item \oportunity: Write access to a callback function pointer.
\end{enumerate} 

The MMO schema is especially useful for vulnerability analysis. For example, we show that without either, a privilege escalation attack is impossible. But, other high impact attacks like a full memory dump may still be viable.  

We build a static code analysis tool to flag high risk device drivers. We use our tool on Linux Kernel 5.3 and identify numerous device drivers at risk for high impact DMA attacks; attacks, like code injection and full memory dump. Our static analysis is able to flag drivers where \means{} or \oportunity{} are present. In the case of I/O devices \motivation{} is usually trivial as the device usually has a legitimate write access. 

\SV{here...}

\textcolor{olive}{Often, the trifecta is present in a single buffer; this makes some DMA attacks trivial\cite{thunder}. This triviality, of known DMA attacks; where a single buggy design choice compromises a system, is actually, a bane to security. The anecdotal nature of these issues, leads to a patchwork of localized fixes rather than a comprehensive solution.} We focus on more complex attacks. Situations, where the MMO trifecta is incomplete. \textcolor{olive}{Analysis of such cases, reveals a dangerous misconception. It is assumed, that buggy device drivers or poor but isolated design choices, are to blame for DMA vulnerabilities.} We demonstrate, that it is often the Kernel it self, that supplements the missing pieces of the trifecta. 

In our attacks, we assume a kernel that utilizes best practices for code injection mitigation i.e KASLR,NX-bit and enables IOMMU by default. The Linux kernel is one such kernel. Linux is the de facto OS of the cloud and runs on most of the worlds servers. The Linux, also has no known DMA attacks against it; though some were theorized\cite{MMT16,thunder}.

Our contributions in this paper are as follows:
\begin{itemize}
    \item MMO Schema for evaluating DMA attacks; we show that the trifecta must be complete for a successful attack. 
    \item We evaluate the Linux kernel device drivers with static code analysis; to find around 500 drivers at high levels of risk.
    \item We demonstrate multiple DMA attacks on the Linux Kernel, where initially the trifecta is incomplete. 
    %We show how the Linux kernel itself; rather than the drivers leak access. These attacks reveal a deep vulnerability in the Linux kernel, de-facto, the OS of the cloud.
    %\item We show how the OS API for the device drives is at fault; rather than the device driver programmers.
    \item We provide a set of actionable advice for better DMA security which is compatible with the performance needs of modern systems.
\end{itemize}